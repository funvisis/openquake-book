OpenQuake offers the possibility to perform different types of analysis/calculations for seismic hazard assessment:
\begin{itemize}
\item Classical (Cornell-type) PSHA
\item Event based PSHA
\item Disaggregation analysis
\item Uniform Hazard Spectra calculation
\item Deterministic (single earthquake scenario) SHA
\end{itemize}

The different types of analysis can be selected inside the \Verb+[general]+ section of the configuration file, by setting  the \Verb+CALCULATION_MODE+ key:
\begin{Verbatim}[frame=single]
[general]

CALCULATION_MODE =
\end{Verbatim}

Valid values are: 
\begin{itemize}
\item \Verb+Classical+ (for Classical PSHA)
\item \Verb+Event Based+ (for Event Based PSHA)
\item \Verb+Disaggregation+ (for Disaggregation analysis)
\item \Verb+UHS+ (for Uniform Hazard Spectra calculation)
\item \Verb+Deterministic+ ( for Deterministic SHA)
\end{itemize}

Each hazard analysis is performed over a set of geographical locations, that can be defined as grid points in a geographical region:

\begin{Verbatim}[frame=single]
REGION_VERTEX = LAT_1, LON_1, LAT_2, LON_2, ..., LAT_N, LON_N
REGION_GRID_SPACING = DELTA_GRID
\end{Verbatim}

where \Verb+REGION_VERTEX+ is a list of vertices coordinates (latitude, longitude) defining a polygonal region. The list of vertices can be defined in clock or counter-clock wise order. \Verb+REGION_GRID_SPACING+ defines the discretization step utilized for the grid construction. No restrictions are given in the number of vertices in the polygon definition.

A second option is to define a list of independent geographical locations:
\begin{Verbatim}[frame=single]
SITES = LAT_1, LON_1, LAT_2, LON_2,..., LAT_N, LON_N
\end{Verbatim}
each location being defined by latitude and longitude. Again no restrictions are given in the number of locations.

Once the \Verb+CALCULATION_MODE+ and the locations of interest have been defined, the user is required to define the input data and the calculation parameters for the requested analysis.

\section{Input Data definition for Classical, Event Based, Disaggregation and UHS analysis}
Input data for probabilistic based seismic hazard analysis (Classical, Event based, Disaggregation, and UHS) are structured in terms of a:
\begin{itemize}
\item file describing the \gls{seismicsourcesystem}
\item file describing the \glspl{groundmotionsystem}
\end{itemize}
The paths to the \gls{seismicsourcesystem} and  \glspl{groundmotionsystem} are given in the 
\Verb+SOURCE_MODEL_LOGIC_TREE_FILE+ and \Verb
+GMPE_LOGIC_TREE_FILE+ keys, respectively.

\subsection{The Seismic Source System definition}

\subsubsection{Seismic source definition in nrML format}
Four source typologies can be currently defined:
\begin{itemize}
\item Area
\item Point
\item Simple Fault
\item Complex Fault
\end{itemize}
Each source typology is identified by a specific element name in the nrML format:
\begin{itemize}
\item \Verb+areaSource+ (for Area)
\item \Verb+pointSource+ (for Point)
\item \Verb+simpleFaultSource+ (for Simple Fault)
\item \Verb+complexFaultSource+ (for Complex Fault)
\end{itemize}
and is characterized by three common attributes:
\begin{itemize}
\item ID: unique identifier 
\item name: source name
\item tectonic region type: tectonic region the source belongs to. 
\end{itemize}
The tectonic region type can be one of five typologies:
\begin{itemize}
\item \Verb+Active Shallow Crust+
\item \Verb+Stable Shallow Crust+
\item \Verb+Subduction Interface+
\item \Verb+Subduction IntraSlab+
\item \Verb+Volcanic+
\end{itemize}
In the nrML format, each source is therefore defined as follows:
\begin{itemize}
\item area:
\begin{Verbatim}[frame=single]
    <areaSource gml:id="SOURCE_ID">
      <gml:name>SOURCE_NAME</gml:name>
      <tectonicRegion>TECTONIC_REGION_TYPE</tectonicRegion>
      ...
\end{Verbatim}
\item point:
\begin{Verbatim}[frame=single]
    <pointSource gml:id="SOURCE_ID">
      <gml:name>SOURCE_NAME</gml:name>
      <tectonicRegion>TECTONIC_REGION_TYPE</tectonicRegion>
      ...
\end{Verbatim}
\item simple fault:
\begin{Verbatim}[frame=single]
    <simpleFaultSource gml:id="SOURCE_ID">
      <gml:name>SOURCE_NAME</gml:name>
      <tectonicRegion>TECTONIC_REGION_TYPE</tectonicRegion>
      ...
\end{Verbatim}
\item complex fault:
\begin{Verbatim}[frame=single]
    <complexFaultSource gml:id="SOURCE_ID">
      <gml:name>SOURCE_NAME</gml:name>
      <tectonicRegion>TECTONIC_REGION_TYPE</tectonicRegion>
      ...
\end{Verbatim}
\end{itemize}
\subsubsection{Area source definition}
An area source can be utilized to describe a polygonal geographical region where seismicity is assumed to be uniform.
The geographic region is defined by the \Verb+areaBoundary+ element. More specifically the area boundary is defined by a list of vertices coordinates (longitude, latitude) in a \Verb+LinearRing+ element defining the exterior boundary of a \Verb+Polygon+ element:
\begin{Verbatim}[frame=single]
    ...
      <areaBoundary>
        <gml:Polygon>
          <gml:exterior>
            <gml:LinearRing>
              <gml:posList>LON_1  LAT_1 LON_2 LAT_2 ... LON_N LAT_N</gml:posList>
            </gml:LinearRing>
          </gml:exterior>
        </gml:Polygon>
      </areaBoundary>
     ...
\end{Verbatim}
\begin{Verbatim}[frame=single]
    <areaSource gml:id="src_1">
      <gml:name>Source 8.CH.3</gml:name>
      <tectonicRegion>Active Shallow Crust</tectonicRegion>
      <areaBoundary>
        <gml:Polygon>
          <gml:exterior>
            <gml:LinearRing>
              <gml:posList>-0.5 -0.5 -0.5 0.5 0.5 0.5 0.5 -0.5</gml:posList>
            </gml:LinearRing>
          </gml:exterior>
        </gml:Polygon>
      </areaBoundary>
      <ruptureRateModel>
        <truncatedGutenbergRichter>
          <aValueCumulative>2.5</aValueCumulative>
          <bValue>0.7318999871612379</bValue>
          <minMagnitude>5.0</minMagnitude>
          <maxMagnitude>8.0</maxMagnitude>
        </truncatedGutenbergRichter>
        <focalMechanism publicID="smi:fm1/0">
          <qml:nodalPlanes>
            <qml:nodalPlane1>
              <qml:strike>
                <qml:value>0.0</qml:value>
              </qml:strike>
              <qml:dip>
                <qml:value>90.0</qml:value>
              </qml:dip>
              <qml:rake>
                <qml:value>0.0</qml:value>
              </qml:rake>
            </qml:nodalPlane1>
          </qml:nodalPlanes>
        </focalMechanism>
      </ruptureRateModel>
      <ruptureDepthDistribution>
        <magnitude>6.5 7.5</magnitude>
        <depth>2.5 0.0</depth>
      </ruptureDepthDistribution>
      <hypocentralDepth>5.0</hypocentralDepth>
    </areaSource>
\end{Verbatim}

\subsection{The Ground Motion System definition}
\begin{comment}
In this Chapter we provide with the reader a description of the input 
files containing the information necessary to completely describe a 
\gls{pshainputmodel} following a format compatible with \gls{acr:oq}. 
%
In OpenQuake the information commonly characterizing a 
\gls{pshainputmodel} is organized into - at least - four main files:
%
\begin{itemize}
	\item A \gls{configurationfile} (usually named config.gem) - It contains the 	
	\item A file describing the \gls{seismicsourcesystem}
	\item One or several files corresponding to the number of 
		\glspl{initialseismicsourcemodel}
	\item A file containing the information relative to the 
		\glspl{groundmotionsystem}
\end{itemize}

%
% ==============================================================================
\section{Anatomy of the configuration file for hazard calculation}
The configuration file contains the following parts
parts\footnote{
Each line starting with a cancel symbol is a comment line that is skipped
by the file parser.
}:
\begin{itemize}
	\item general 
	\item hazard
\end{itemize}
Each line starting with a cancel symbol is a comment line that is skipped
by the file parser. 

In order to discuss the content of the input file we use the 'config.gem' 
provided with the hazard demo relative to the PEER test set 1 case 10.

This is the general section of this file 
\begin{Verbatim}[baselinestretch=1,fontsize=\small,numbers=left,frame=single]
[general]
CALCULATION_MODE = Classical
# NOTE: The order of the vertices is to be kept!!!
# lat, lon of polygon vertices (in clock or counter-clock wise order)
REGION_VERTEX = 38.000, -122.000, 38.000, -122.000, 38.000, ...
# degrees
REGION_GRID_SPACING = 0.1
\end{Verbatim}
%
The first line contains a label specifying the section (in this 
particular case the 'general' section.

The second part composing the configuration file contains hazard 
specific information that can be divided into the following units: 
general calculation parameters and PSHA input model files, 
ground-motion model, seismic source model section and, results.
%
%
\subsubsection{General calculation parameters and PSHA input model files}
%
\begin{Verbatim}[baselinestretch=1,fontsize=\small,numbers=left,frame=single]
[HAZARD]
SOURCE_MODEL_LT_RANDOM_SEED = 23
GMPE_LT_RANDOM_SEED = 5
GMF_RANDOM_SEED = 3

# file containing erf logic tree structure
SOURCE_MODEL_LOGIC_TREE_FILE = source_model_logic_tree.xml
# file containing gmpe logic tree structure
GMPE_LOGIC_TREE_FILE = gmpe_logic_tree.xml
# output directory - relative to this file
OUTPUT_DIR = computed_output

# moment magnitude (Mw)
MINIMUM_MAGNITUDE = 5.0
# years
INVESTIGATION_TIME = 1.0
# maximum integration distance (km)
MAXIMUM_DISTANCE = 200.0
# bin width of the magnitude frequency distribution
WIDTH_OF_MFD_BIN = 0.1
\end{Verbatim}

%
% ============================================================================== 
\section{Description of the seismic source system file}
The seismic source system is described by an XML formatted file 
compatible with the NRML schema.
\marginpar{We miss a citation for NRML and a website as well!!}
%
\begin{Verbatim}[baselinestretch=1,fontsize=\small,numbers=left,frame=single]
	<logicTreeSet>
        <logicTree id="lt1">
            <logicTreeBranchSet branchingLevel="1" 
                                uncertaintyType="sourceModel">
                <logicTreeBranch>
                    <uncertaintyModel>source_model.xml</uncertaintyModel>
                    <uncertaintyWeight>1.0</uncertaintyWeight>
                </logicTreeBranch>
             </logicTreeBranchSet>
        </logicTree>
    </logicTreeSet>
\end{Verbatim}
%
% ------------------------------------------------------------------------------
\subsection{Basic seismic source typologies }

%
% ------------------------------------------------------------------------------
\subsection{Initial Seismic Source Model description}
The initial seismic source model used in this example is extremely simple, 
since it contains a single area source. 
%
\begin{Verbatim}[baselinestretch=1,fontsize=\small,numbers=left,frame=single]
    <areaSource gml:id="Src1">
      <gml:name>PEER test "AREA 1" model</gml:name>
      <tectonicRegion>Active Shallow Crust</tectonicRegion>
      <areaBoundary>
        <gml:Polygon>
          <gml:exterior>
            <gml:LinearRing>
              <gml:posList>
				-122.000 38.901 
				-121.920 38.899 
				...
				-122.160 38.892 
				-122.080 38.899 
			  </gml:posList>
            </gml:LinearRing>
          </gml:exterior>
        </gml:Polygon>
      </areaBoundary>
      <ruptureRateModel>
        <truncatedGutenbergRichter>
          <aValueCumulative>3.097</aValueCumulative>
          <bValue>0.9</bValue>
          <minMagnitude>5.0</minMagnitude>
          <maxMagnitude>6.5</maxMagnitude>
        </truncatedGutenbergRichter>
        <focalMechanism publicID="smi:f01/11">
          <qml:nodalPlanes>
            <qml:nodalPlane1>
              <qml:strike>
                <qml:value>0.0</qml:value>
              </qml:strike>
              <qml:dip>
                <qml:value>90.0</qml:value>
              </qml:dip>
              <qml:rake>
                <qml:value>0.0</qml:value>
              </qml:rake>
            </qml:nodalPlane1>
          </qml:nodalPlanes>
        </focalMechanism>
      </ruptureRateModel>
      <ruptureDepthDistribution>
        <magnitude>6.5</magnitude>
        <depth>5.0</depth>
      </ruptureDepthDistribution>
      <hypocentralDepth>5.0</hypocentralDepth>
    </areaSource>
  </sourceModel>
\end{Verbatim}


%
% ============================================================================== 
\section{Ground-motion system file description}
\end{comment}