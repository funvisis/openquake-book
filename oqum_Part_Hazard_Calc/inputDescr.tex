\section{Input Data definition for Classical, Event Based, Disaggregation and UHS}
Input data for probabilistic based seismic hazard analysis (Classical, Event based, Disaggregation, and UHS) are structured in terms of a:
\begin{itemize}
\item file describing the Seismic Source System, that is the set of initial source models and associated epistemic uncertainties needed to model the seismic activity in the region of interest.
\item file describing the Ground Motion System, that is the set of ground motion prediction equations, per tectonic region type, needed to model the ground motion shaking in the region of interest.
\end{itemize}
The paths to the Seismic Source and Ground Motion System files are given in the 
\Verb+SOURCE_MODEL_LOGIC_TREE_FILE+ and \Verb
+GMPE_LOGIC_TREE_FILE+ keys, respectively, both defined in the \Verb+[HAZARD]+ section of the configuration file.\\
The information required to define both the Seismic Source and Ground Motion Systems is stored in a \Verb+logicTree+ element: 
\begin{Verbatim}[frame=single, commandchars=\\\{\}]
<\textcolor{red}{logicTree} logicTreeID="ID">
...
</\textcolor{red}{logicTree}>
\end{Verbatim}
A \Verb+logicTree+ is defined as a sequence of \Verb+logicTreeBranchingLevel+ elements. The position in the sequence specifies in which level of the tree the branching level is located. That is, the first logicTreeBranchingLevel element in the sequence represents the first level in the tree, the second element the second level in the tree, and so on.
\begin{Verbatim}[frame=single, commandchars=\\\{\}]
<\textcolor{red}{logicTree} logicTreeID="ID">
	<\textcolor{green}{logicTreeBranchingLevel} branchingLevelID="ID_1">
		...
	</\textcolor{green}{logicTreeBranchingLevel}>
	<\textcolor{green}{logicTreeBranchingLevel} branchingLevelID="ID_2">
		...
	</\textcolor{green}{logicTreeBranchingLevel}>
	....
	<\textcolor{green}{logicTreeBranchingLevel} branchingLevelID="ID_N">
		...
	</\textcolor{green}{logicTreeBranchingLevel}>
</\textcolor{red}{logicTree}>
\end{Verbatim}
No restrictions are present on the number of tree levels that can be defined.\\
A \Verb+logicTreeBranchingLevel+ is defined as a sequence of \Verb+logicTreeBranchSet+ elements. Each \Verb+logicTreeBranchSet+ defines a particular epistemic uncertainty inside a branching level. A branch set has two required attributes (\Verb+branchingLevelID+ and \Verb+uncertaintyType+ (defining the type of epistemic uncertainty the branch set is defining))
\begin{Verbatim}[frame=single, commandchars=\\\{\}]
<\textcolor{red}{logicTree} logicTreeID="ID">
...
	<\textcolor{green}{logicTreeBranchingLevel} branchingLevelID="ID_#">
		<\textcolor{blue}{logicTreeBranchSet} branchSetID="ID_1"
			uncertaintyType="UNCERTAINTY_TYPE">
			...
		</\textcolor{blue}{logicTreeBranchSet}>
		<\textcolor{blue}{logicTreeBranchSet} branchSetID="ID_2"
			uncertaintyType="UNCERTAINTY_TYPE">
			...
		</\textcolor{blue}{logicTreeBranchSet}>
		...
		<\textcolor{blue}{logicTreeBranchSet} branchSetID="ID_N"
			uncertaintyType="UNCERTAINTY_TYPE">
			...
		</\textcolor{blue}{logicTreeBranchSet}>
	</\textcolor{green}{logicTreeBranchingLevel}>
...
</\textcolor{red}{logicTree}>
\end{Verbatim}
Possible values for the \Verb+uncertaintyType+ attribute are:
\begin{itemize}
\item \Verb+gmpeModel+: identifying epistemic uncertainties on ground motion prediction equations
\item \Verb+sourceModel+: identifying epistemic uncertainties on source models
\item \Verb+maxMagGRRelative+: identifying epistemic uncertainties (relative: that is increments) to be added (or subtracted, depending on the sign of the increment) to the Gutenberg-Richter maximum magnitude value.
\item \Verb+bGRRelative+: identifying epistemic uncertainties (relative) to be applied to the Gutenberg-Richter b value.
\item \Verb+abGRAbsolute+:identifying epistemic uncertainties (absolute: that is new values used to replace original values) on the Gutenberg-Richter a and b values.
\item \Verb+maxMagGRAbsolute+: identifying epistemic uncertainties (absolute) on the Gutenberg-Richter maximum magnitude.
\end{itemize}
No restrictions are given on the number of branch sets that can be defined inside a branching level.\\
A \Verb+branchSet+ is defined as a sequence of  \Verb+logicTreeBranch+ elements, each defined by an \Verb+uncertaintyModel+ element (a string identifying an uncertainty model; the content of the string varies with the uncertaintyType attribute value of the branchSet element) and the uncertaintyWeight element (specifying the probability/weight associated to the uncertaintyModel):
\begin{Verbatim}[frame=single, commandchars=\\\{\}]
<\textcolor{red}{logicTree} logicTreeID="ID">
...
	<\textcolor{green}{logicTreeBranchingLevel} branchingLevelID="ID_#">
		...
		<\textcolor{blue}{logicTreeBranchSet} branchSetID="ID_#"
				uncertaintyType="UNCERTAINTY_TYPE">
			<\textcolor{magenta}{logicTreeBranch} branchID="ID_1">
				<uncertaintyModel>
				UNCERTAINTY_MODEL
				</uncertaintyModel>
				<uncertaintyWeight>
				UNCERTAINTY_WEIGHT
				</uncertaintyWeight>
			</\textcolor{magenta}{logicTreeBranch}>
			...
			<\textcolor{magenta}{logicTreeBranch} branchID="ID_N">
				<uncertaintyModel>
				UNCERTAINTY_MODEL
				</uncertaintyModel>
				<uncertaintyWeight>
				UNCERTAINTY_WEIGHT
				</uncertaintyWeight>
			</\textcolor{magenta}{logicTreeBranch}>
		</\textcolor{blue}{logicTreeBranchSet}>
		...
	</\textcolor{green}{logicTreeBranchingLevel}>
...
</\textcolor{red}{logicTree}>
\end{Verbatim}
Depending on the \Verb+uncertaintyType+ the content of the \Verb+<uncertaintyModel>+ element changes:
\begin{itemize}
\item if \Verb+uncertaintyType="gmpeModel"+, the uncertainty model contains the name of a ground motion prediction equation (a list of available GMPEs are given in appendix A), e.g.:
\begin{Verbatim}[frame=single, commandchars=\\\{\}]
<uncertaintyModel>GMPE_NAME</uncertaintyModel>
\end{Verbatim}
\item if \Verb+uncertaintyType="sourceModel"+, the uncertainty model contains the paths to a source model file, e.g.:
\begin{Verbatim}[frame=single, commandchars=\\\{\}]
<uncertaintyModel>SOURCE_MODEL_FILE_PATH</uncertaintyModel>
\end{Verbatim}
\item if \Verb+uncertaintyType="maxMagGRRelative"+, the uncertainty model contains the increment to be added (or subtracted, depending on the sign) to the Gutenberg-Richter maximum magnitude:
\begin{Verbatim}[frame=single, commandchars=\\\{\}]
<uncertaintyModel>MAX_MAGNITUDE_INCREMENT</uncertaintyModel>
\end{Verbatim}
\item if \Verb+uncertaintyType="bGRRelative"+, the uncertainty model contains the increment to be added (or subtracted, depending on the sign) to the Gutenberg-Richter b value:
\begin{Verbatim}[frame=single, commandchars=\\\{\}]
<uncertaintyModel>B_VALUE_INCREMENT</uncertaintyModel>
\end{Verbatim}
\item if \Verb+uncertaintyType="abGRAbsolute"+, the uncertainty model contains one (if the uncertainty apply to a source with only one Gutenberg-Richter magnitude frequency distribution) or more (if the source has more than one magnitude frequency distributions) a and b pairs:
\begin{Verbatim}[frame=single, commandchars=\\\{\}]
<uncertaintyModel>
A_VALUE_1 B_VALUE_1
 ... 
A_VALUE_N B_VALUE_N
</uncertaintyModel>
\end{Verbatim}
\item if \Verb+uncertaintyType="maxMagGRAbsolute"+, the uncertainty model contains one or more (depending on the number of magnitude frequency distributions in the source) Gutenberg-Richter maximum magnitude values:
\begin{Verbatim}[frame=single, commandchars=\\\{\}]
<uncertaintyModel>
MAX_MAGNITUDE_1
 ... 
MAX_MAGNITUDE_N
</uncertaintyModel>
\end{Verbatim}
\end{itemize}
No restrictions are given on the number of \Verb+logicTreeBranch+ elements that can be defined in a \Verb+logicTreeBranchSet+, as long as the uncertainty weights sum to 1.0.\\
The \Verb+logicTreeBranchSet+ element offers also a number of optional attributes allowing for complex tree definitions:
\begin{itemize}
\item \Verb+applyToBranches+: specifies to which \Verb+logicTreeBranch+ elements (one or more), in the previous branching level, the branch set is linked to. The linking is established by defining the IDs of the branches to link to:
\begin{Verbatim}[frame=single, commandchars=\\\{\}]
applyToBranches="logicTreeBranchID1 logicTreeBranchID2 .... logicTreeBranchIDN"
\end{Verbatim}
The default is the keyword ALL, which means that a branch set is by default linked to all branches in the previous branching level. By specifying one or more branches to which the branch set links to, non-symmetric logic trees can be defined.
\item \Verb+applyToSources+: specifies to which source in a source model the uncertainty applies to. Sources are specified in terms of their IDs:
\begin{Verbatim}[frame=single, commandchars=\\\{\}]
applyToSources="srcID1 srcID2 .... srcIDN"
\end{Verbatim}
\item \Verb+applyToSourceType+: specifies to which source type the uncertainty applies to.
Only one source typology can be defined (\Verb+area+, \Verb+point+, \Verb+simpleFault+, \Verb+complexFault+), e.g.:
\begin{Verbatim}[frame=single, commandchars=\\\{\}]
applyToSources="area"
\end{Verbatim}
\item \Verb+applyToTectonicRegionType+: specifies to which tectonic region type the uncertainty applies to. Only one tectonic region type can be defined (\Verb+Active Shallow Crust+, \Verb+Stable Shallow Crust+,  \Verb+Subduction Interface+, \Verb+Subduction IntraSlab+, \Verb+Volcanic+), e.g.:
\begin{Verbatim}[frame=single, commandchars=\\\{\}]
applyToTectonicRegionType="Active Shallow Crust"
\end{Verbatim}
\end{itemize}

\subsection{The Seismic Source System definition}
The Seismic Source System is defined in a \Verb+logicTreeElement+ with the following constrains:
\begin{itemize}
\item The first branching level can contain only one branch set. This branch set must define uncertainties in the source model (that is \Verb+uncertaintyType="sourceModel"+)
\item Subsequent branching levels can define uncertainties of type: \Verb+maxMagGRRelative+, \Verb+bGRRelative+, \Verb+abGRAbsolute+, \Verb+maxMagGRAbsolute+. Relative uncertainties are applied assuming conservation of total moment rate.
\item In all branching levels but the first, only one optional attribute (\Verb+applyToSources+, \Verb+applyToSourceType+, \Verb+applyToTectonicRegionType+) can be defined. The first branching level does not accept any optional attribute.
\item If in a branching level, a branch set has \Verb+applyToBranches="ALL"+, then this branch set must be unique in the branching level. In other words, only one branch set with \Verb+applyToBranches="ALL"+ is allowed per branching level.
\item If in a branching level, more than one branch sets are defined, the attribute \Verb+applyToBranches+ must contain one or more branch IDs referring to branches that are only in the previous branching level and that are not already linked to other branch sets (in other words there cannot be two branch sets that link to the same branch).
\item In all branch sets, branch weights must sum to 1.
\end{itemize}
Figure \ref{lt1} shows an example of Seismic Source System consisting of two initial source models defined in the first branching level. In the second branching level, absolute uncertainties on Gutenberg-Richter a and b values are defined for source model 1 (\Verb+applyToBranches="_11"+). These uncertainties apply only to \Verb+area+ sources (applyToSourceType="area"). Same type of uncertainties are defined also for source model 2 (\Verb+applyToBranches="_12"+) but apply only to sources with IDs equal to \Verb+"_2"+ and \Verb+"_3"+ (\Verb+applyToSources="_2 _3"+). In the third branching level, absolute uncertainties on Gutenberg-Richter maximum magnitude are defined. These uncertainties apply to all source models in the previous branching level (the attribute \Verb+applyToBranches+ is absent, the default value \Verb+"ALL"+ is therefore assumed) but only to those sources that belongs to active shallow crust\\ (\Verb+applyToTectonicRegionType="Active Shallow Crust"+).
\begin{figure}[htbp]
\begin{center}
\begin{Verbatim}[frame=single, commandchars=\\\{\},fontsize=\scriptsize, samepage=true]
<\textcolor{red}{logicTree} logicTreeID="ID">
	<\textcolor{green}{logicTreeBranchingLevel} branchingLevelID="ID">
		<\textcolor{blue}{logicTreeBranchSet} branchSetID="ID"
		  uncertaintyType="\textcolor{orange}{sourceModel}">
			<\textcolor{magenta}{logicTreeBranch} branchID="_11">
			   <uncertaintyModel>SOURCE_MODEL_FILE_PATH</uncertaintyModel>
			   <uncertaintyWeight>WEIGHT</uncertaintyWeight>
			 </\textcolor{magenta}{logicTreeBranch}> 
			<\textcolor{magenta}{logicTreeBranch} branchID="_12">
			  <uncertaintyModel>SOURCE_MODEL_FILE_PATH</uncertaintyModel>
			  <uncertaintyWeight>WEIGHT</uncertaintyWeight>
			</\textcolor{magenta}{logicTreeBranch}>
		</\textcolor{blue}{logicTreeBranchSet}>
	</\textcolor{green}{logicTreeBranchingLevel}>
	<\textcolor{green}{logicTreeBranchingLevel} branchingLevelID="ID">
		<\textcolor{blue}{logicTreeBranchSet} branchSetID="ID" 
		    uncertaintyType="\textcolor{orange}{abGRAbsolute}"
		    applyToBranches="_11" 
		    applyToSourceType="area">
			<\textcolor{magenta}{logicTreeBranch} branchID="_11_12_21">
			   <uncertaintyModel>A_VALUE B_VALUE</uncertaintyModel>
			   <uncertaintyWeight>WEIGHT</uncertaintyWeight>
			</\textcolor{magenta}{logicTreeBranch}>
			<\textcolor{magenta}{logicTreeBranch} branchID="_11_12_22">
			   <uncertaintyModel>A_VALUE B_VALUE</uncertaintyModel>
			   <uncertaintyWeight>WEIGHT</uncertaintyWeight>
			</\textcolor{magenta}{logicTreeBranch}>  
		</\textcolor{blue}{logicTreeBranchSet}>
		<\textcolor{blue}{logicTreeBranchSet} branchSetID="ID" 
		   uncertaintyType="\textcolor{orange}{abGRAbsolute}"
	            applyToBranches="_12" 
		   applyToSources="_2 _3">
			<\textcolor{magenta}{logicTreeBranch} branchID="_13_21">
			   <uncertaintyModel>A_VALUE B_VALUE</uncertaintyModel>
			   <uncertaintyWeight>WEIGHT</uncertaintyWeight>
			   </\textcolor{magenta}{logicTreeBranch}>
			<\textcolor{magenta}{logicTreeBranch} branchID="_13_22">
			   <uncertaintyModel>A_VALUE B_VALUE</uncertaintyModel>
			   <uncertaintyWeight>WEIGHT</uncertaintyWeight>
			</\textcolor{magenta}{logicTreeBranch}>
		</\textcolor{blue}{logicTreeBranchSet}>
	</\textcolor{green}{logicTreeBranchingLevel}>
	<\textcolor{green}{logicTreeBranchingLevel} branchingLevelID="ID">
		<\textcolor{blue}{logicTreeBranchSet} branchSetID="ID" 
		   uncertaintyType="\textcolor{orange}{maxMagGRAbsolute}" 
		   applyToTectonicRegionType="Active Shallow Crust">
			<\textcolor{magenta}{logicTreeBranch} branchID="_31">
			   <uncertaintyModel>MAXIMUM_MAGNITUDE</uncertaintyModel>
			   <uncertaintyWeight>WEIGHT</uncertaintyWeight>
			</\textcolor{magenta}{logicTreeBranch}>
			<\textcolor{magenta}{logicTreeBranch} branchID="_32">
			   <uncertaintyModel>MAXIMUM_MAGNITUDE</uncertaintyModel>
			   <uncertaintyWeight>WEIGHT</uncertaintyWeight>
			</\textcolor{magenta}{logicTreeBranch}>
		</\textcolor{blue}{logicTreeBranchSet}>
	</\textcolor{green}{logicTreeBranchingLevel}>
</\textcolor{red}{logicTree}>
\end{Verbatim}
\caption{Example of Seismic Source System definition. Two source models are considered, and epistemic uncertainties (absolute) on Gutenberg-Richter a and b, and maximum magnitude are defined.}
\label{lt1}
\end{center}
\end{figure}

\subsection{The Seismic Source Model definition}\label{seismicSourceNrml}
A Seismic Source Model is defined as a collection of seismic sources. Four source typologies can be currently defined:
\begin{itemize}
\item Area
\item Point
\item Simple Fault
\item Complex Fault
\end{itemize}
Each source typology is identified by a specific element name in the nrML format:
\begin{itemize}
\item \Verb+areaSource+ (for Area)
\item \Verb+pointSource+ (for Point)
\item \Verb+simpleFaultSource+ (for Simple Fault)
\item \Verb+complexFaultSource+ (for Complex Fault)
\end{itemize}
Regardless of the source typology three common attributes must be defined:
\begin{itemize}
\item ID
\item name
\item tectonic region type 
\end{itemize}
\begin{Verbatim}[frame=single, commandchars=\\\{\},fontsize=\normalsize, samepage=true]
<\textcolor{red}{sourceModel} gml:id="ID">
	...
	<\textcolor{green}{areaSource} gml:id="SOURCE_ID">
		<gml:name>SOURCE_NAME</gml:name>
		<tectonicRegion>TECT_REGION_TYPE</tectonicRegion>
		...
	</\textcolor{green}{areaSource}>
	...
	<\textcolor{green}{pointSource} gml:id="SOURCE_ID">
		<gml:name>SOURCE_NAME</gml:name>
		<tectonicRegion>TECT_REGION_TYPE</tectonicRegion>
		...
	</\textcolor{green}{pointSource}>
	...
	<\textcolor{green}{simpleFaultSource} gml:id="SOURCE_ID">
		<gml:name>SOURCE_NAME</gml:name>
		<tectonicRegion>TECT_REGION_TYPE</tectonicRegion>
		...
	</\textcolor{green}{simpleFaultSource}>
	...
	<\textcolor{green}{complexFaultSource} gml:id="SOURCE_ID">
		<gml:name>SOURCE_NAME</gml:name>
		<tectonicRegion>TECT_REGION_TYPE</tectonicRegion>
		...
	</\textcolor{green}{complexFaultSource}>
	...
</\textcolor{red}{sourceModel}>
\end{Verbatim}
A single source model can therefore contain sources of different typologies, and no constrains are given on the number of sources that can be defined in a source model.

\subsubsection{Area source definition}
An area source can be utilized to describe a polygonal geographical region where seismicity is assumed to be uniform. The area source-specific elements are:
\begin{itemize}
\item \Verb+areaBoundary+: defines the area boundary.
\item \Verb+ruptureRateModel+: defines a MFD-Focal Mechanism pair.
\item \Verb+ruptureDepthDistribution+: defines the (top of) rupture depth distribution versus magnitude.
\item \Verb+hypocentralDepth+: defines ruptures' hypocentral depth.
\end{itemize}
More then one \Verb+ruptureRateModels+ can be defined in a single area source (giving the possibility to define specific MFDs for specific focal mechanisms).
\begin{Verbatim}[frame=single, commandchars=\\\{\},fontsize=\normalsize, samepage=true]
<\textcolor{red}{areaSource} gml:id="ID">
	<gml:name>NAME</gml:name>
	<tectonicRegion>TECT_REG_TYPE</tectonicRegion>
	<\textcolor{green}{areaBoundary}>
		...
	</\textcolor{green}{areaBoundary}>
	<\textcolor{blue}{ruptureRateModel}>
		...
	</\textcolor{blue}{ruptureRateModel}>
	<\textcolor{blue}{ruptureRateModel}>
		...
	</\textcolor{blue}{ruptureRateModel}>
	...
	...
	<\textcolor{blue}{ruptureRateModel}>
		...
	</\textcolor{blue}{ruptureRateModel}>
	<\textcolor{magenta}{ruptureDepthDistribution}>
		...
	</\textcolor{magenta}{ruptureDepthDistribution}>
	<\textcolor{orange}{hypocentralDepth}>
		...	
	</\textcolor{orange}{hypocentralDepth}>
</\textcolor{red}{areaSource}>
\end{Verbatim}
The area boundary is defined by a list of vertices coordinates (longitude, latitude) in a \Verb+LinearRing+ element defining the exterior boundary of a \Verb+Polygon+ element:
\begin{Verbatim}[frame=single, commandchars=\\\{\},fontsize=\normalsize, samepage=true]
<\textcolor{red}{areaBoundary}>
	<gml:\textcolor{green}{Polygon}>
		<gml:\textcolor{blue}{exterior}>
			<gml:\textcolor{magenta}{LinearRing}>
				<gml:\textcolor{orange}{posList}>
					LON_1 LAT_1
					LON_2 LAT_2
						...			
					LON_N LAT_N
				</gml:\textcolor{orange}{posList}>
			</gml:\textcolor{magenta}{LinearRing}>
		</gml:\textcolor{blue}{exterior}>
	</gml:\textcolor{green}{Polygon}>
</\textcolor{red}{areaBoundary}>
\end{Verbatim}
The \Verb+ruptureRateModel+ element contains the definition of a magnitude frequency distribution (\Verb+truncatedGutenbergRichter+ or \Verb+evenlyDiscretizedIncrementalMFD+) and a focal mechanism (\Verb+focalMechanism+). That is:
\begin{Verbatim}[frame=single, commandchars=\\\{\},fontsize=\normalsize, samepage=true]
<\textcolor{red}{ruptureRateModel}>
	<\textcolor{green}{truncatedGutenbergRichter}>
		...
	</\textcolor{green}{truncatedGutenbergRichter}>
	<\textcolor{blue}{focalMechanism}>
		...
	</\textcolor{blue}{focalMechanism}>
</\textcolor{red}{ruptureRateModel}>
\end{Verbatim}
or:
\begin{Verbatim}[frame=single, commandchars=\\\{\},fontsize=\normalsize, samepage=true]
<\textcolor{red}{ruptureRateModel}>
	<\textcolor{green}{evenlyDiscretizedIncrementalMFD}>
		...
	</\textcolor{green}{evenlyDiscretizedIncrementalMFD}>
	<\textcolor{blue}{focalMechanism}>
		...
	</\textcolor{blue}{focalMechanism}>
</\textcolor{red}{ruptureRateModel}>
\end{Verbatim}
A truncated Gutenberg-Richter magnitude frequency distribution is defined in terms of minimum and maximum magnitudes, cumulative a value, and b value:
\begin{Verbatim}[frame=single, commandchars=\\\{\},fontsize=\normalsize, samepage=true]
<\textcolor{red}{truncatedGutenbergRichter}>
	<\textcolor{green}{aValueCumulative}>CUMULATIVE_A_VALUE</\textcolor{green}{aValueCumulative}>
	<\textcolor{blue}{bValue}>B_VALUE</\textcolor{blue}{bValue}>
	<\textcolor{magenta}{minMagnitude}>MINIMUM_MAGNITUDE</\textcolor{magenta}{minMagnitude}>
	<\textcolor{orange}{maxMagnitude}>MAXIMUM_MAGNITUDE</\textcolor{orange}{maxMagnitude}>
</\textcolor{red}{truncatedGutenbergRichter}>
\end{Verbatim}
The incremental evenly-discretized magnitude frequency distribution is defined instead as:
\begin{Verbatim}[frame=single, commandchars=\\\{\},fontsize=\normalsize, samepage=true]
<\textcolor{red}{evenlyDiscretizedIncrementalMFD} 
        \textcolor{green}{binSize}="BIN_SIZE"
 	 \textcolor{blue}{minVal}="MINIMUM_VALUE">
              RATE_1
              RATE_2
              ...
              RATE_N
</\textcolor{red}{evenlyDiscretizedIncrementalMFD}>
\end{Verbatim}
where \Verb+BIN_SIZE+ represents the discretization interval of the magnitude-frequency distribution, and \Verb+MINIMUM_VALUE+ represents the minimum value in the distribution (interpreted as the middle-point value of the first bin in the distribution).\\
The focal mechanism is defined in terms of strike, dip and rake values:
\begin{Verbatim}[frame=single, commandchars=\\\{\},fontsize=\normalsize, samepage=true]
<\textcolor{red}{focalMechanism} publicID="ID">
	<qml:\textcolor{green}{nodalPlanes}>
		<qml:\textcolor{blue}{nodalPlane1}>
			<qml:\textcolor{magenta}{strike}>
			<qml:value>STRIKE_VALUE</qml:value>
			</qml:\textcolor{magenta}{strike}>
			<qml:\textcolor{orange}{dip}>
			<qml:value>DIP_VALUE</qml:value>
			</qml:\textcolor{orange}{dip}>
			<qml:\textcolor{violet}{rake}>
			<qml:value>RAKE_VALUE</qml:value>
			</qml:\textcolor{violet}{rake}>
		</qml:\textcolor{blue}{nodalPlane1}>
	</qml:\textcolor{green}{nodalPlanes}>
</\textcolor{red}{focalMechanism}>
 \end{Verbatim}
where
\begin{alltt} \(0\degree \leq\) STRIKE_VALUE \(\leq 360\degree\) \end{alltt}
\begin{alltt} \(0\degree \leq\) DIP_VALUE \(\leq 90\degree\) \end{alltt}
\begin{alltt} \(-180\degree \leq\) RAKE_VALUE \(\leq +180\degree\) \end{alltt}
The \Verb+ruptureDepthDistribution+ contains a list of magnitude and depth values:
\begin{Verbatim}[frame=single, commandchars=\\\{\},fontsize=\normalsize, samepage=true]
<\textcolor{red}{ruptureDepthDistribution}>
	<\textcolor{green}{magnitude}>MAG_1 MAG_2 ... MAG_N</\textcolor{green}{magnitude}>
	<\textcolor{blue}{depth}>DEPTH_1 DEPTH_2 ... DEPTH_N</\textcolor{blue}{depth}>
</\textcolor{red}{ruptureDepthDistribution}>
\end{Verbatim}
The last element needed to define an area source is the hypocentral depth:
\begin{Verbatim}[frame=single, commandchars=\\\{\},fontsize=\normalsize, samepage=true]
<\textcolor{red}{hypocentralDepth}>HYPOCENTRAL_DEPTH</\textcolor{red}{hypocentralDepth}>
\end{Verbatim}
that is expressed by a single value.

\subsubsection{Point source definition}
The point source definition follows the same schema of the area source definition, the only
difference being the absence of the \Verb+areaBoundary+ element, that is replaced by a
\Verb+location+ element defining the coordinates of a single geographic site:
\begin{Verbatim}[frame=single, commandchars=\\\{\},fontsize=\normalsize, samepage=true]
<\textcolor{red}{pointSource} gml:id="ID">
	<gml:name>NAME</gml:name>
	<tectonicRegion>TECT_REG_TYPE</tectonicRegion>
	<\textcolor{green}{location}>
		...
	</\textcolor{green}{location}>
	<\textcolor{blue}{ruptureRateModel}>
		...
	</\textcolor{blue}{ruptureRateModel}>
	<\textcolor{blue}{ruptureRateModel}>
		...
	</\textcolor{blue}{ruptureRateModel}>
	...
	...
	<\textcolor{blue}{ruptureRateModel}>
		...
	</\textcolor{blue}{ruptureRateModel}>
	<\textcolor{magenta}{ruptureDepthDistribution}>
		...
	</\textcolor{magenta}{ruptureDepthDistribution}>
	<\textcolor{orange}{hypocentralDepth}>
		...	
	</\textcolor{orange}{hypocentralDepth}>
</\textcolor{red}{areaSource}>
\end{Verbatim}
where the \Verb+location+ element is defined as follows:
\begin{Verbatim}[frame=single, commandchars=\\\{\},fontsize=\normalsize, samepage=true]
<\textcolor{red}{location}>
	<gml:\textcolor{green}{Point}>
	<gml:pos>LON LAT</gml:pos>
	</gml:\textcolor{green}{Point}>
</\textcolor{red}{location}>
\end{Verbatim}

\subsubsection{Simple fault source definition}
The fault source-specific elements are:
\begin{itemize}
\item \Verb+rake+: defines the rake value.
\item \Verb+truncatedGutenbergRichter+ or \Verb+evenlyDiscretizedIncrementalMFD+: define the magnitude frequency distribution.
\item \Verb+simpleFaultGeometry+: defines the fault surface geometry.
\end{itemize}
\begin{Verbatim}[frame=single, commandchars=\\\{\},fontsize=\normalsize, samepage=true]
<\textcolor{red}{simpleFaultSource} gml:id="ID">
	<gml:name>SOURCE_NAME</gml:name>
	<tectonicRegion>TECT_REG_TYPE</tectonicRegion>
	<\textcolor{green}{rake}>
	...
	</\textcolor{green}{rake}>
	<\textcolor{blue}{truncatedGutenbergRichter}>
	...
	</\textcolor{blue}{truncatedGutenbergRichter}>
		\textit{OR}
	<\textcolor{blue}{evenlyDiscretizedIncrementalMFD}>
	...
	</\textcolor{blue}{evenlyDiscretizedIncrementalMFD}>
	<\textcolor{magenta}{simpleFaultGeometry}>
	...
	</\textcolor{magenta}{simpleFaultGeometry}>
</\textcolor{red}{simpleFaultSource}>
\end{Verbatim}
The \Verb+rake+ element contains a single rake value:
\begin{Verbatim}[frame=single, commandchars=\\\{\},fontsize=\normalsize, samepage=true]
<\textcolor{red}{rake}>RAKE_VALUE</\textcolor{red}{rake}>
\end{Verbatim}
The \Verb+simpleFaultGeometry+ element is defined as follows:
\begin{Verbatim}[frame=single, commandchars=\\\{\},fontsize=\normalsize, samepage=true]
<\textcolor{red}{simpleFaultGeometry} gml:id="ID">
	<\textcolor{green}{faultTrace}>
		<gml:\textcolor{blue}{LineString}>
			<gml:posList>
				LON1 LAT1 DEPTH1
				LON2 LAT2 DEPTH2
				...
				LONN LATN DEPTHN
			</gml:posList>
		</gml:\textcolor{blue}{LineString}>
	</\textcolor{green}{faultTrace}>
	<\textcolor{magenta}{dip}>DIP_VALUE</\textcolor{magenta}{dip}>
	<\textcolor{orange}{upperSeismogenicDepth}>UPPER_DEPTH</\textcolor{orange}{upperSeismogenicDepth}>
	<\textcolor{violet}{lowerSeismogenicDepth}>LOWER_DEPTH</\textcolor{violet}{lowerSeismogenicDepth}>
</\textcolor{red}{simpleFaultGeometry}>
\end{Verbatim}
%%%%%%%%%%%%%%%%%%%%%%%%%%%%%%%%%%%%%%%
\subsection{The Ground Motion System definition}
The Ground Motion System is defined in a \Verb+logicTree+ element with the following constrains:
\begin{itemize}
\item Only one branch set can be defined per branching level.
\item Each branch set must define uncertainties of type \Verb+gmpeModel+.
\item All branch sets must define the applyToTectonicRegionType attribute. This is the only attribute allowed.
\item Each branch set must refer to a different tectonic region type.
\item In all branch sets, branch weights must sum to 1.
\end{itemize}
Figure \ref{lt2} shows an example of Ground Motion System as defined in a \Verb+logicTreeElement+. Each branching level contains only one \Verb+branchSet+, each referring to a different tectonic region type. Each branch set can contain a variable number of GMPEs.
\begin{figure}[htbp]
\begin{center}
\begin{Verbatim}[frame=single, commandchars=\\\{\},fontsize=\scriptsize, samepage=true]
<\textcolor{red}{logicTree} logicTreeID="ID">       
	<\textcolor{green}{logicTreeBranchingLevel} branchingLevelID="ID">
		<\textcolor{blue}{logicTreeBranchSet} branchSetID="ID" uncertaintyType="gmpeModel" 
		applyToTectonicRegionType="\textcolor{orange}{Active Shallow Crust}">
			<\textcolor{magenta}{logicTreeBranch} branchID="ID">
				<uncertaintyModel>GMPE_NAME</uncertaintyModel>
				<uncertaintyWeight>WEIGHT</uncertaintyWeight>
			</\textcolor{magenta}{logicTreeBranch}>
			<\textcolor{magenta}{logicTreeBranch} branchID="ID">
				<uncertaintyModel>GMPE_NAME</uncertaintyModel>
				<uncertaintyWeight>WEIGHT</uncertaintyWeight>
			</\textcolor{magenta}{logicTreeBranch}>                
		</\textcolor{blue}{logicTreeBranchSet}>       
	</\textcolor{green}{logicTreeBranchingLevel}>
		<\textcolor{green}{logicTreeBranchingLevel} branchingLevelID="ID">
		<\textcolor{blue}{logicTreeBranchSet} branchSetID="ID" uncertaintyType="gmpeModel" 
		applyToTectonicRegionType="\textcolor{orange}{Stable Shallow Crust}">
			<\textcolor{magenta}{logicTreeBranch} branchID="ID">
				<uncertaintyModel>GMPE_NAME</uncertaintyModel>
				<uncertaintyWeight>WEIGHT</uncertaintyWeight>
			</\textcolor{magenta}{logicTreeBranch}>
			<\textcolor{magenta}{logicTreeBranch} branchID="ID">
				<uncertaintyModel>GMPE_NAME</uncertaintyModel>
				<uncertaintyWeight>WEIGHT</uncertaintyWeight>
			</\textcolor{magenta}{logicTreeBranch}>                
		</\textcolor{blue}{logicTreeBranchSet}>       
	</\textcolor{green}{logicTreeBranchingLevel}>
		<\textcolor{green}{logicTreeBranchingLevel} branchingLevelID="ID">
		<\textcolor{blue}{logicTreeBranchSet} branchSetID="ID" uncertaintyType="gmpeModel" 
		applyToTectonicRegionType="\textcolor{orange}{Subduction IntraSlab}">
			<\textcolor{magenta}{logicTreeBranch} branchID="ID">
				<uncertaintyModel>GMPE_NAME</uncertaintyModel>
				<uncertaintyWeight>WEIGHT</uncertaintyWeight>
			</\textcolor{magenta}{logicTreeBranch}>
			<\textcolor{magenta}{logicTreeBranch} branchID="ID">
				<uncertaintyModel>GMPE_NAME</uncertaintyModel>
				<uncertaintyWeight>WEIGHT</uncertaintyWeight>
			</\textcolor{magenta}{logicTreeBranch}>                
		</\textcolor{blue}{logicTreeBranchSet}>       
	</\textcolor{green}{logicTreeBranchingLevel}>
	<\textcolor{green}{logicTreeBranchingLevel} branchingLevelID="ID">
		<\textcolor{blue}{logicTreeBranchSet} branchSetID="ID" uncertaintyType="gmpeModel" 
		applyToTectonicRegionType="\textcolor{orange}{Subduction Interface}">
			<\textcolor{magenta}{logicTreeBranch} branchID="ID">
				<uncertaintyModel>GMPE_NAME</uncertaintyModel>
				<uncertaintyWeight>WEIGHT</uncertaintyWeight>
			</\textcolor{magenta}{logicTreeBranch}>                
		</\textcolor{blue}{logicTreeBranchSet}>
	</\textcolor{green}{logicTreeBranchingLevel}>
	<\textcolor{green}{logicTreeBranchingLevel} branchingLevelID="ID">
		<\textcolor{blue}{logicTreeBranchSet} branchSetID="ID" uncertaintyType="gmpeModel" 
		applyToTectonicRegionType="\textcolor{orange}{Volcanic}">
			<\textcolor{magenta}{logicTreeBranch} branchID="ID">
				<uncertaintyModel>GMPE_NAME</uncertaintyModel>
				<uncertaintyWeight>WEIGHT</uncertaintyWeight>
			</\textcolor{magenta}{logicTreeBranch}>                
		</\textcolor{blue}{logicTreeBranchSet}>
	</\textcolor{green}{logicTreeBranchingLevel}>           
</\textcolor{red}{logicTree}>
\end{Verbatim}
\caption{Example of Ground Motion System definition. GMPEs are defined for the tectonic region types currently supported by OpenQuake.}
\label{lt2}
\end{center}
\end{figure}


\begin{comment}
In this Chapter we provide with the reader a description of the input 
files containing the information necessary to completely describe a 
\gls{pshainputmodel} following a format compatible with \gls{acr:oq}. 
%
In OpenQuake the information commonly characterizing a 
\gls{pshainputmodel} is organized into - at least - four main files:
%
\begin{itemize}
	\item A \gls{configurationfile} (usually named config.gem) - It contains the 	
	\item A file describing the \gls{seismicsourcesystem}
	\item One or several files corresponding to the number of 
		\glspl{initialseismicsourcemodel}
	\item A file containing the information relative to the 
		\glspl{groundmotionsystem}
\end{itemize}

%
% ==============================================================================
\section{Anatomy of the configuration file for hazard calculation}
The configuration file contains the following parts
parts\footnote{
Each line starting with a cancel symbol is a comment line that is skipped
by the file parser.
}:
\begin{itemize}
	\item general 
	\item hazard
\end{itemize}
Each line starting with a cancel symbol is a comment line that is skipped
by the file parser. 

In order to discuss the content of the input file we use the 'config.gem' 
provided with the hazard demo relative to the PEER test set 1 case 10.

This is the general section of this file 
\begin{Verbatim}[baselinestretch=1,fontsize=\small,numbers=left,frame=single]
[general]
CALCULATION_MODE = Classical
# NOTE: The order of the vertices is to be kept!!!
# lat, lon of polygon vertices (in clock or counter-clock wise order)
REGION_VERTEX = 38.000, -122.000, 38.000, -122.000, 38.000, ...
# degrees
REGION_GRID_SPACING = 0.1
\end{Verbatim}
%
The first line contains a label specifying the section (in this 
particular case the 'general' section.

The second part composing the configuration file contains hazard 
specific information that can be divided into the following units: 
general calculation parameters and PSHA input model files, 
ground-motion model, seismic source model section and, results.
%
%
\subsubsection{General calculation parameters and PSHA input model files}
%
\begin{Verbatim}[baselinestretch=1,fontsize=\small,numbers=left,frame=single]
[HAZARD]
SOURCE_MODEL_LT_RANDOM_SEED = 23
GMPE_LT_RANDOM_SEED = 5
GMF_RANDOM_SEED = 3

# file containing erf logic tree structure
SOURCE_MODEL_LOGIC_TREE_FILE = source_model_logic_tree.xml
# file containing gmpe logic tree structure
GMPE_LOGIC_TREE_FILE = gmpe_logic_tree.xml
# output directory - relative to this file
OUTPUT_DIR = computed_output

# moment magnitude (Mw)
MINIMUM_MAGNITUDE = 5.0
# years
INVESTIGATION_TIME = 1.0
# maximum integration distance (km)
MAXIMUM_DISTANCE = 200.0
# bin width of the magnitude frequency distribution
WIDTH_OF_MFD_BIN = 0.1
\end{Verbatim}

%
% ============================================================================== 
\section{Description of the seismic source system file}
The seismic source system is described by an XML formatted file 
compatible with the NRML schema.
\marginpar{We miss a citation for NRML and a website as well!!}
%
\begin{Verbatim}[baselinestretch=1,fontsize=\small,numbers=left,frame=single]
	<logicTreeSet>
        <logicTree id="lt1">
            <logicTreeBranchSet branchingLevel="1" 
                                uncertaintyType="sourceModel">
                <logicTreeBranch>
                    <uncertaintyModel>source_model.xml</uncertaintyModel>
                    <uncertaintyWeight>1.0</uncertaintyWeight>
                </logicTreeBranch>
             </logicTreeBranchSet>
        </logicTree>
    </logicTreeSet>
\end{Verbatim}
%
% ------------------------------------------------------------------------------
\subsection{Basic seismic source typologies }

%
% ------------------------------------------------------------------------------
\subsection{Initial Seismic Source Model description}
The initial seismic source model used in this example is extremely simple, 
since it contains a single area source. 
%
\begin{Verbatim}[baselinestretch=1,fontsize=\small,numbers=left,frame=single]
    <areaSource gml:id="Src1">
      <gml:name>PEER test "AREA 1" model</gml:name>
      <tectonicRegion>Active Shallow Crust</tectonicRegion>
      <areaBoundary>
        <gml:Polygon>
          <gml:exterior>
            <gml:LinearRing>
              <gml:posList>
				-122.000 38.901 
				-121.920 38.899 
				...
				-122.160 38.892 
				-122.080 38.899 
			  </gml:posList>
            </gml:LinearRing>
          </gml:exterior>
        </gml:Polygon>
      </areaBoundary>
      <ruptureRateModel>
        <truncatedGutenbergRichter>
          <aValueCumulative>3.097</aValueCumulative>
          <bValue>0.9</bValue>
          <minMagnitude>5.0</minMagnitude>
          <maxMagnitude>6.5</maxMagnitude>
        </truncatedGutenbergRichter>
        <focalMechanism publicID="smi:f01/11">
          <qml:nodalPlanes>
            <qml:nodalPlane1>
              <qml:strike>
                <qml:value>0.0</qml:value>
              </qml:strike>
              <qml:dip>
                <qml:value>90.0</qml:value>
              </qml:dip>
              <qml:rake>
                <qml:value>0.0</qml:value>
              </qml:rake>
            </qml:nodalPlane1>
          </qml:nodalPlanes>
        </focalMechanism>
      </ruptureRateModel>
      <ruptureDepthDistribution>
        <magnitude>6.5</magnitude>
        <depth>5.0</depth>
      </ruptureDepthDistribution>
      <hypocentralDepth>5.0</hypocentralDepth>
    </areaSource>
  </sourceModel>
\end{Verbatim}


%
% ============================================================================== 
\section{Ground-motion system file description}
\end{comment}