The OQ \gls{pshainputmodel} consists of the following files:
\begin{itemize}
	\item A configuration file (usually named config.gem)	
	\item A file describing the \gls{seismicsourcesystem}
	\item One or several files corresponding to the number of 
		\glspl{initialseismicsourcemodel}
	\item A file containing the information relative to the 
		\glspl{groundmotionsystem}
\end{itemize}
%
% ==============================================================================
\section{Anatomy of the Configuration file for hazard calculation}
The configuration file contains the following parts:
\begin{itemize}
	\item general 
	\item hazard
\end{itemize}
Each line starting with a cancel symbol is a comment line that is skipped
by the file parser. 

In order to discuss the content of the input file we use the 'config.gem' 
provided with the hazard demo relative to the PEER test set 1 case 10.

This is the general section of this file 
\begin{Verbatim}[baselinestretch=1,fontsize=\small,numbers=left,frame=single]
[general]
CALCULATION_MODE = Classical
# NOTE: The order of the vertices is to be kept!!!
# lat, lon of polygon vertices (in clock or counter-clock wise order)
REGION_VERTEX = 38.000, -122.000, 38.000, -122.000, 38.000, ...
# degrees
REGION_GRID_SPACING = 0.1
\end{Verbatim}
%
The first line contains a label specifying the section (in this 
particular case the 'general' section.

The second part composing the configuration file contains hazard 
specific information that can be divided into the following units: 
general calculation parameters and PSHA input model files, 
ground-motion model, seismic source model section and, results.
%
%
\subsubsection{General calculation parameters and PSHA input model files}
%
\begin{Verbatim}[baselinestretch=1,fontsize=\small,numbers=left,frame=single]
[HAZARD]
SOURCE_MODEL_LT_RANDOM_SEED = 23
GMPE_LT_RANDOM_SEED = 5
GMF_RANDOM_SEED = 3

# file containing erf logic tree structure
SOURCE_MODEL_LOGIC_TREE_FILE = source_model_logic_tree.xml
# file containing gmpe logic tree structure
GMPE_LOGIC_TREE_FILE = gmpe_logic_tree.xml
# output directory - relative to this file
OUTPUT_DIR = computed_output

# moment magnitude (Mw)
MINIMUM_MAGNITUDE = 5.0
# years
INVESTIGATION_TIME = 1.0
# maximum integration distance (km)
MAXIMUM_DISTANCE = 200.0
# bin width of the magnitude frequency distribution
WIDTH_OF_MFD_BIN = 0.1
\end{Verbatim}

%
% ============================================================================== 
\section{Description of the seismic source system file}
The seismic source system is described by an XML formatted file 
compatible with the NRML schema.
\marginpar{We miss a citation for NRML and a website as well!!}
%
\begin{Verbatim}[baselinestretch=1,fontsize=\small,numbers=left,frame=single]
	<logicTreeSet>
        <logicTree id="lt1">
            <logicTreeBranchSet branchingLevel="1" 
                                uncertaintyType="sourceModel">
                <logicTreeBranch>
                    <uncertaintyModel>source_model.xml</uncertaintyModel>
                    <uncertaintyWeight>1.0</uncertaintyWeight>
                </logicTreeBranch>
             </logicTreeBranchSet>
        </logicTree>
    </logicTreeSet>
\end{Verbatim}
%
% ------------------------------------------------------------------------------
\subsection{Basic seismic source typologies }

%
% ------------------------------------------------------------------------------
\subsection{Initial Seismic Source Model description}
The initial seismic source model used in this example is extremely simple, 
since it contains a single area source. 
%
\begin{Verbatim}[baselinestretch=1,fontsize=\small,numbers=left,frame=single]
    <areaSource gml:id="Src1">
      <gml:name>PEER test "AREA 1" model</gml:name>
      <tectonicRegion>Active Shallow Crust</tectonicRegion>
      <areaBoundary>
        <gml:Polygon>
          <gml:exterior>
            <gml:LinearRing>
              <gml:posList>
				-122.000 38.901 
				-121.920 38.899 
				...
				-122.160 38.892 
				-122.080 38.899 
			  </gml:posList>
            </gml:LinearRing>
          </gml:exterior>
        </gml:Polygon>
      </areaBoundary>
      <ruptureRateModel>
        <truncatedGutenbergRichter>
          <aValueCumulative>3.097</aValueCumulative>
          <bValue>0.9</bValue>
          <minMagnitude>5.0</minMagnitude>
          <maxMagnitude>6.5</maxMagnitude>
        </truncatedGutenbergRichter>
        <focalMechanism publicID="smi:f01/11">
          <qml:nodalPlanes>
            <qml:nodalPlane1>
              <qml:strike>
                <qml:value>0.0</qml:value>
              </qml:strike>
              <qml:dip>
                <qml:value>90.0</qml:value>
              </qml:dip>
              <qml:rake>
                <qml:value>0.0</qml:value>
              </qml:rake>
            </qml:nodalPlane1>
          </qml:nodalPlanes>
        </focalMechanism>
      </ruptureRateModel>
      <ruptureDepthDistribution>
        <magnitude>6.5</magnitude>
        <depth>5.0</depth>
      </ruptureDepthDistribution>
      <hypocentralDepth>5.0</hypocentralDepth>
    </areaSource>
  </sourceModel>
\end{Verbatim}


%
% ============================================================================== 
\section{Ground-motion system file description}