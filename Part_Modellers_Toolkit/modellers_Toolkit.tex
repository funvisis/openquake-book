%
% ------------------------------------------------------------------------------
\section{Seismic catalogue processing}

%
%  . . . . . . . . . . . . . . . . . . . . . . . . . . . . . . . . . . . . . . . 
\subsection{Catalogue declustering}

%
%  . . . . . . . . . . . . . . . . . . . . . . . . . . . . . . . . . . . . . . . 
\subsection{Accounting for uncertainties}
Instrumental catalogues:
\begin{itemize}
	\item Magnitude binning (see \citet{felzer2008});
	\item Magnitude uncertainties (see \citet{felzer2008}).
	\item Magnitude completeness (see ).
\end{itemize}
Historical catalogues:
\begin{itemize}
	\item Magnitude binning 
	\item Magnitude uncertainties 
	\item Magnitude completeness 
\end{itemize}
%
% ------------------------------------------------------------------------------
\section{Definition of the seismicity occurence model}
The definition of the seismicity occurrence model for a source requires the 
definition of position, geometry and spatial distribution of seismicity. In 
addition the occurrence model asks for the temporal occurrence model and a
frequency-magnitude distribution.
%
%  . . . . . . . . . . . . . . . . . . . . . . . . . . . . . . . . . . . . . . . 
\subsection{Seismic source geometry definition}

%
%  . . . . . . . . . . . . . . . . . . . . . . . . . . . . . . . . . . . . . . . 
\subsection{Seismic source temporal occurrence model definition}

%
%  . . . . . . . . . . . . . . . . . . . . . . . . . . . . . . . . . . . . . . . 
\subsection{Seismic source frequency magnitude distribution definition}

