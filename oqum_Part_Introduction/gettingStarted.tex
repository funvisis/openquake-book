\section{Prerequisites}
The present manual provides instructions about using OpenQuake to perform seismic hazard and risk analysis. However, no attempts are made to explain the "science" behind the software. The scientific basis and methodologies adopted in the software are illustrated in a separate document, the 'OpenQuake Book':\\ \\
\href{http://openquake.org/wp-content/uploads/2011/10/OpenQuake-Book_Version0.1.pdf}
   {http://openquake.org/wp-content/uploads/2011/10/OpenQuake-Book-Version0.1.pdf}\\ \\
We refer the reader to the OpenQuake Book for a detailed explanation of the algorithms utilized, and indeed we strongly suggest its reading as a prerequisite for utilizing and understanding the present manual.\\
OpenQuake currently runs through a Linux command line interface (CLI), so familiarity with Unix/Linux shell base commands (change directory, list files in a directory, move and copy files...) is required. Base instructions about how to use a CLI through a terminal can be found here: \\ \\
\href{https://help.ubuntu.com/community/UsingTheTerminal}
   {https://help.ubuntu.com/community/UsingTheTerminal}\\

\section{Installing OpenQuake}
The OpenQuake installation is currently supported on the Ubuntu Operating System (\href{http://www.ubuntu.com/}
   {http://www.ubuntu.com/}), and installation instructions can be find here:\\ \\
    \href{https://github.com/gem/openquake/wiki/Ubuntu-11.04}
   {https://github.com/gem/openquake/wiki/Ubuntu-11.04}\\ \\
The installation instructions refer to a single-user/ single-machine scenario. Installation for a multi-user / multi-machine 
environment is currently not supported. To checking that OpenQuake is installed and working properly, see the section 'Using the engine'.\\
Installation on Mac OS X and Windows is not supported, however a workaround is possible by using a virtual machine. For instruction see here:\\ \\
\href{https://github.com/gem/openquake/wiki/Mac-OS-X}
   {https://github.com/gem/openquake/wiki/Mac-OS-X}   for Mac\\ \\
   and here:\\ \\
   \href{https://github.com/gem/openquake/wiki/Windows-OS}
   {https://github.com/gem/openquake/wiki/Windows-OS} for Windows.\\ \\
As an alternative of locally installing OpenQuake, the GEM-Model Facility has developed the OpenQuake Alpha Testing Service that gives the possibility to access the latest OpenQuake Alpha release on a cloud based server:\\ \\
\href{http://openquake.org/alpha-testing-services/}
   {http://openquake.org/alpha-testing-services/}
   
\section{Running OpenQuake}
OpenQuake runs through a CLI. The software can be launched by typing the command \Verb+openquake+. Typing the command alone displays a short description of the OpenQuake functionalities and a number of 'flags' for option specifications:
\begin{Verbatim}[frame=single, commandchars=\\\{\}, fontsize=\scriptsize]
user@users-desktop:~$ openquake
OpenQuake: software for seismic hazard and risk assessment

It receives its inputs through a configuration file plus input data in .xml
format and stores the results in .xml format.

Available Hazard Analysis

  Classical PSHA
    Input   Source Model Logic Tree
            GMPE Logic Tree

    Output  Hazard maps
            Hazard curves

  Event-Based PSHA
    Input   Source Model Logic Tree
            GMPE Logic Tree

    Output  Ground Motion fields

  Deterministic SHA
    Input   Rupture Model

    Output  Ground Motion fields

Available Risk Analysis

  Classical PSHA-based
    Input   Exposure (a value per asset)
            Vulnerability curves (a list of vulnerability functions)
            Seismic hazard input: hazard curves

    Output  A grid of loss-ratio curves
            A grid of loss curves
            A map of losses at each interval

  Probabilistic event-based
    Input   Exposure (a value per asset)
            Vulnerability curves (a list of vulnerability functions)
            Seismic hazard input: sets of ground motion fields

    Output  A grid of loss-ratio curves
            A grid of loss curves
            A map of losses at each interval
            An aggregated loss curve


Flags:
  --config_file: OpenQuake configuration file
    (default: 'openquake-config.gem')
  --debug: Turns on debug logging and verbose output. One of debug, info, warn,
    error, critical.
    (default: 'warn')
  --help: Show this help
    (default: 'false')
  --output_type: <db|xml>: Computation result output type
    (default: 'db')
  --version: Show version information
    (default: 'false')
\end{Verbatim}
OpenQuake is instructed through a configuration file (whose path is provided by the \Verb+--config_file+ flag). The configuration file contains informations about what type of analysis to perform, what input data to use, and associated calculations parameters. Input data must be provided into XML formatted files following the "NRML" schema: \\ \\
 \href{https://github.com/gem/openquake/tree/master/openquake/nrml/schema}
{https://github.com/gem/openquake/tree/master/openquake/nrml/schema}\\ \\ 
By default OpenQuake saves output data into the OpenQuake-Data Base (that is included in the OpenQuake package). To have output data saved into (XML formatted) files the option \Verb+output_type=xml+ must be specified.
\begin{Verbatim}[frame=single, commandchars=\\\{\}, samepage=true]
openquake --\textcolor{red}{config_file}=/PATH/TO/CONFIG/FILE --\textcolor{red}{output_type}=xml
\end{Verbatim}

\section{The configuration file}
The configuration file follows the INI file format, and consist of statements in the format \Verb+KEY=VALUE+ organized in three main sections:
\begin{Verbatim}[frame=single, commandchars=\\\{\}, samepage=true]
[\textcolor{red}{general}]
...
[\textcolor{green}{HAZARD}]
...
[\textcolor{blue}{RISK}]
...
\end{Verbatim}
 
The \Verb+[general]+ section contains calculation parameters that are common to both hazard and risk analysis, while the \Verb+[HAZARD]+ and \Verb+[RISK]+ sections contains parameters specific for hazard and risk analysis, respectively.

The different types of analysis can be selected inside the \Verb+[general]+ section, by setting  the \Verb+CALCULATION_MODE+ key:
\begin{Verbatim}[frame=single, commandchars=\\\{\}, samepage=true]
[\textcolor{red}{general}]

CALCULATION_MODE =
\end{Verbatim}

Valid values are: 
\begin{itemize}
\item \Verb+Classical+ (for Classical Probabilistic Seismic Hazard/ Risk Analysis)
\item \Verb+Event Based+ (for Event Based Probabilistic Seismic Hazard/ Risk Analysis)
\item \Verb+Disaggregation+ (for Disaggregation analysis)
\item \Verb+UHS+ (for Uniform Hazard Spectra calculation)
\item \Verb+Deterministic+ ( for Deterministic Seismic Hazard/ Risk Analysis)
\end{itemize}

Each analysis is performed over a set of geographical locations, that can be defined as grid points in a geographical region:

\begin{Verbatim}[frame=single, commandchars=\\\{\}, samepage=true]
[\textcolor{red}{general}]
...
REGION_VERTEX = LAT_1, LON_1, LAT_2, LON_2, ..., LAT_N, LON_N
REGION_GRID_SPACING = DELTA_GRID
\end{Verbatim}

where \Verb+REGION_VERTEX+ is a list of vertices coordinates (latitude, longitude) defining a polygonal region. The list of vertices can be defined in clock or counter-clock wise order. \Verb+REGION_GRID_SPACING+ defines the discretization step utilized for the grid construction. No restrictions are given in the number of vertices in the polygon definition.

A second option is to define a list of independent geographical locations:
\begin{Verbatim}[frame=single, commandchars=\\\{\}, samepage=true]
[\textcolor{red}{general}]
...
SITES = LAT_1, LON_1, LAT_2, LON_2,..., LAT_N, LON_N
\end{Verbatim}
each location being defined by latitude and longitude. Again no restrictions are given in the number of locations.

In case of risk analysis, a third option allows to perform calculations on the geographical locations defined in the exposure model. To select this option, the key \Verb+COMPUTE_HAZARD_AT_ASSETS_LOCATIONS+ must be set to \Verb+true+ in the \Verb+[RISK]+ section of the configuration file.
\begin{Verbatim}[frame=single, commandchars=\\\{\}, samepage=true]
[\textcolor{red}{RISK}]
...
COMPUTE_HAZARD_AT_ASSETS_LOCATIONS = true
\end{Verbatim}
if set to false, the region or sites defined in the \Verb+[general]+ section will be used instead.

Once the \Verb+CALCULATION_MODE+ and the locations of interest have been defined, the user is required to define the input data and the calculation parameters (in the configuration file) for the requested analysis.

Chapters \ref{chap:hazinp} and \ref{chap:riskinp} describe the parameters and input data needed to perform all the supported hazard an risk analysis.