This book aims to provide an explanation of the scientific basis 
and the methodologies adopted in the implementation of OpenQuake, an open 
source code for seismic hazard and risk calculation. 
%
The book follows the traditional openness and transparency features of the 
\gls{acr:gem} as clearly indicated in the development principles of 
OpenQuake. 

%
The \gls{acr:gem} initiative aims at establishing uniform, open 
standards to calculate and communicate earthquake risk worldwide, by 
developing together with the community a global, state-of-the-art and 
dynamic earthquake risk model. 
%
OpenQuake is a fully integrated, flexible and scalable hazard and risk 
calculation engine whose development is at the core of \gls{acr:gem}'s
overall objectives.
% ------------------------------------------------------------------------------
\section{The Basics of OpenQuake}
The implementation of OpenQuake officially started in Summer 2010 
following the experience gained in \gls{acr:gem}'s kick-off project GEM1 
\citep{gemfoundation2010}, during which an extensive appraisal of existing hazard 
and risk codes was performed \citep{danciu2010,crowley2010}
and prototype hazard and risk software were selected, designed and
implemented \citep{pagani2010,crowley2010a}.

Currently OpenQuake is a blend of Java and Python code developed 
following the most common requirements of Open Source 
software development, such as a public repository, IRC channel and open mailing lists. 
The source code, released under an open source software license,
is freely and openly accessible on a web based repository 
(see \href{http://github.com/gem}{github.com/gem}) while the 
development process is managed so that the community can participate 
to the day by day development as well as in the mid- and long-term 
design process. 
%
OpenQuake development also leverages from a number of open source projects 
such as \href{http://www.opensha.org}{OpenSHA}, \href {http://celeryproject.org}{Celeryd} and \href{http://www.rabbitmq.com}{RabbitMQ}, just to mention a few.

The hazard component of the engine, which constitutes almost entirely
the Java part of the code, largely relies on classes belonging to 
\gls{opensha}, a comprehensive library for performing state-or-the-art
PSHA developed collaboratively at the \gls{acr:usgs} and at the 
\gls{acr:scec}. New code was developed in GEM1 and in the following 
phases to support a standardized \gls{acr:erf} and logic-tree structure, 
event-based PSHA and, seismic hazard disaggregation.
%
The Risk component of the engine was designed in GEM1, prototyped in Java and eventually
coded in Python by the OpenQuake team operating at the \gls{acr:gem} 
Model Facility.

A schema that illustrates OpenQuake's structure is 
represented in Figure \ref{fig:openquake_schema}; the schema contains:
purple boxes representing the main modules of the hazard component, 
green boxes showing the modules of the risk component, white boxes
with main outputs computed by the distinct modules and orange rectangles
displaying the main input information that should be entered into the calculation engine. 
%
% ------------------------------------------------------------------------------
%\subsection{Brief description of the OpenQuake IT architecture}
%\input{./Part_Introduction/it.tex}

%
% ------------------------------------------------------------------------------
\section{Book structure}
The OpenQuake book is organized into three parts; in the first we give
a broad introduction to OpenQuake and the Book, in the second we 
describe the science behind the hazard component of the engine
whilst in the third we illustrate the theory of the risk calculators
incorporated into OpenQuake.

\hfill \\
\emph{Part II: Hazard}
\begin{itemize}
\item Chapter \ref{chap:inthaz} offers an introduction to the hazard 
topics discussed in the following chapters. In particular, in this 
Chapter we discuss the main OpenQuake concepts and we illustrate the 
calculation workflows currently available in the hazard component of 
OpenQuake.
\item Chapter \ref{chap:hazinp} focuses on the structure and the 
characteristics of the information necessary to define a 
comprehensive PSHA input model. This Chapter also includes
descriptions of the main seismic source typologies and of the logic 
tree structure.
\item We dedicate Chapter \ref{chap:erf} to the explanation of the
methodology adopted for the processing of the logic tree structures
supported by OpenQuake and for the creation of the 
\gls{earthquakeruptureforecast}
\item The last Chapter of the hazard part (chapter \ref{chap:hazcalc}) 
illustrates the main calculators available: the classical-PSHA calculator,
the event-based calculator and the disaggregation calculator. 
\end{itemize}
\emph{Part III: Risk}
\begin{itemize}
\item Chapter \ref{chap:intrisk} introduces the main risk concepts and workflows. 
\item Chapter \ref{chap:riskinput} contains an explanation of the exposure, 
physical vulnerability and fragility concepts.
\item Chapter \ref{chap:risk_deterministic} describes the deterministic event-based risk 
methodology implemented in \gls{acr:oq}.
\item Chapter \ref{chap:risk_prob_event_based} provides an overview of the 
probabilistic event-based risk calculation methodology.
\item Chapter \ref{chap:risk_psha_based} illustrates risk calculations based on 
the hazard curves from classical PSHA.
\end{itemize}
%
In the closing part, the Book contains a glossary that aims to define a 
clear and unique terminology.
%
% . . . . . . . . . . . . . . . . . . . . . . . . . . . . . . . . . . . > Figure
\begin{landscape}
\begin{figure}
\includegraphics[width=20cm,angle=0]{./Figures/Part_Introduction/engine9_20110130.eps}
\caption{OpenQuake schema. Purple boxes are the calculators included in the  
the hazard part of OQ; green boxes are the risk calculators. The method of 
\citet{wesson2009} (not currently supported) is represented in a separate box since it incorporates 
hazard and risk calculations.}
\label{fig:openquake_schema}
\end{figure}
\end{landscape}
% . . . . . . . . . . . . . . . . . . . . . . . . . . . . . . . . . . . < Figure