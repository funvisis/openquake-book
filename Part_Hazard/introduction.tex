Probabilistic Seismic Hazard is nowadays a well established methodology, largely founded on the works of \citeauthor{cornell1968} and \citeauthor{esteva1968}, both published at the end of the 1960's. 
%
The development of PSHA within the latest four decades did not change much the original concept but made calculations more rigorous and accurate, especially with respect to treatment of uncertainties. 
%
The evolution of PSHA methodologies proceeded in parallel with the development of instrumental seismology and computing hardware. USGS and UNAM covered an important role in the development of PSHA. Computer codes such as EQRISK \citep{mcguire1976} and different SEISRISK versions \citep{bender1982,bender1987} traced the advancement of PSHA calculation within
the last part of the 20th century.
% 
Nowadays the most computationally intensive PSHA models available are the ones developed for site-specific PSHA analyses, such as the ones performed for special installations, and the regional PSHA input models. In the first case most of the demand comes from the complexity of the input whilst in the second case is the number of sites considered that makes calculation heavy.  
%
% ------------------------------------------------------------------------------
\section{OpenQuake-hazard}
OpenQuake-hazard leverages from OpenSHA (http://www.opensha.org) - an open-source, Java-based platform for conducting Seismic Hazard Analysis - and it is developed in collaboration with the OpenSHA team. 