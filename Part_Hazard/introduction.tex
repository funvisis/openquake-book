Probabilistic Seismic Hazard is nowadays a well established methodology, largely founded on the works of \citeauthor{cornell1968} and \citeauthor{esteva1968}, both published at the end of the 1960's. 
%
The development of PSHA within the latest four decades did not change much the original concept but made calculations more rigorous and accurate, especially with respect to the treatment of uncertainties. 

The evolution of PSHA methodologies proceeded in parallel with the development of instrumental seismology and hardware computing power. Computer codes such as EQRISK \citep{mcguire1976} and different SEISRISK versions \citep{bender1982,bender1987} traced the advancement of PSHA calculation within
the last part of the 20th century.

At the present time, the most computationally intensive PSHA models available are the ones developed for site-specific PSHA analyses, such as the ones performed for special installations, and the regional PSHA input models. In the first case most of the computation demand comes from the complexity of the input whilst in the second case is the number of sites considered that makes calculations particularly heavy.  
%
% ------------------------------------------------------------------------------
\section{OpenQuake-hazard: main concepts}
OpenQuake-hazard leverages from OpenSHA (http://www.opensha.org) - an open-source, Java-based platform for conducting Seismic Hazard Analysis - and it is developed in collaboration with the OpenSHA team. 

Schematically, the procedure that OpenQuake follows to compute the hazard is the following:
%
\begin{enumerate}
%
\item \emph{Read the PSHA input model - i.e. the union of the Seismic Sources System and the Ground Motion Prediction Equations System - and calculation settings.}
	\index{Seismic Sources!System} %%%%%%
	\index{PSHA!Input model} %%%%%%
	\index{Ground Motion! Prediction Equations!System} %%%%%%
	%
	The \emph{Seismic Sources System} is an object that contains the information necessary to create one or several Seismic Sources Model, eventually by taking into account the epistemic uncertainties. 
	%
	In particular, the Seismic Sources System contains:
	\begin{itemize}
	\item One or several \emph{Initial Seismic Sources Models};
	\index{Logic Tree!Seismic sources} %%%%%%
	\item One logic tree - the Seismic Sources Logic Tree - describing epistemic uncertainties connected with the objects and parameters characterizing the Initial Seismic Sources Models.
	\end{itemize}
	%
	The GMPEs System is an object that contains the information necessary to create one or several GMPE model, eventually by taking into account the epistemic uncertainties. 
	\begin{itemize}
	\item One or several Ground Motion Prediction Equations;
	\index{Logic Tree!Ground Motion Prediction Equations} %%%%%%
	\item One logic tree - the Ground Motion Prediction Equations Logic Tree - describing epistemic uncertainties connected with the objects and parameters characterizing the selected GMPEs.	
	\end{itemize}

%
\item \emph{Process the logic tree structures to account for epistemic uncertainties connected with the seismic sources and the ground motion prediction equations and, create Seismic Sources Models and Ground Motion Prediction Equations Models}.
	\index{Seismic Sources!Model} %%%%%%
	\index{Ground Motion! Prediction Equations!Model} %%%%%%
	%
	
	A Seismic Sources Model contains the information necessary to create an Earthquake Rupture Forecast (i.e. the probabilistic seismicity occurrence model) without considering any epistemic uncertainty.
	
	A Ground Motion Prediction Equations Model includes the information necessary to compute hazard using a Seismic Sources Model. 

\item Compute the hazard considering as many Seismic Sources Models and Ground Motion Prediction Equations Models as need to adequately characterize uncertainties.
\item Post-process the results obtained for distinct calculations.
\end{enumerate}
%