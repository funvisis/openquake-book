Seismic source characterization consist on defining: 
\begin{enumerate}
\item The frequency and size distibution of earthquakes;
\item The geometry of earthquake ruptures;
\item The maximum magnitude;
\item The spatial location of future earthquakes.
\end{enumerate}
In the following sections we give an overview of the probabilistic models 
available and the methods to describe and characterize 
%
% ------------------------------------------------------------------------------
\newpage
\section{Frequency and size distribution of earthquakes}
The definition of frequency and size distribution of earthquakes inevitably 
involves the need to process seismic catalogues. 
%
%  - - - - - - - - - - - - - - - - - - - - - - - - - - - - - - - - - - - - - - -
\subsection{Catalogues pre-processing}
%
%  - - - - - - - - - - - - - - - - - - - - - - - - - - - - - - - - - - - - - - -
\subsubsection{Catalogue declustering}
\begin{itemize}
\item \cite{gardner74}
\item \cite{reasenberg85}
\item \cite{kagan10}
\end{itemize}
%
%  - - - - - - - - - - - - - - - - - - - - - - - - - - - - - - - - - - - - - - -
\subsection{Deriving parameters descriptive of the G-R relationship}
\begin{itemize}
\item \cite{weichert80}
See also CALCRATE \citep{bender87,hanson92}
\end{itemize}
%
% --------------------------------------------------------------------------------
\newpage
\section{M$_{\text{max}}$ definition}
According to \cite{kijko04} two are the main approaches available to calculate 
$m_{max}$ for a given seismic source or region defined deterministic and 
probabilistic, respectively.
%
In some cases, particularly in the cases where a probabilistic methodology is used, 
the definition of $m_{max}$ can be interrelated with the definition of seismicity 
parameters.
%
%  - - - - - - - - - - - - - - - - - - - - - - - - - - - - - - - - - - - - - - - -
\subsubsection{EPRI method  \citep{johnston94}}  
??
%
%  - - - - - - - - - - - - - - - - - - - - - - - - - - - - - - - - - - - - - - - -
\subsection{Deterministic methods}
%
%  - - - - - - - - - - - - - - - - - - - - - - - - - - - - - - - - - - - - - - - -
\subsection{Probabilistic methods}
\subsubsection{Kijko method \citep{kijko04}} 

%
% --------------------------------------------------------------------------------
\newpage
\section{Spatial location of future earthquakes and geometry of earthquake 
ruptures}
In the PEGASOS project, in some cases, $m_{max}$, $b_{GR}$ values, and rate of 
events per unit of area were considered variable within an area. This idea however
lacks of a clear definition of the area source concept.

\newpage
% --------------------------------------------------------------------------------
\section{Integrating distinct information layers}
%  - - - - - - - - - - - - - - - - - - - - - - - - - - - - - - - - - - - - - - - -
\subsection{Crustal strain rate, geology and seismicity}
%  . . . . . . . . . .  . . . . . . . . . . . . . . . . . . . . . . . . . . . . . . 
\subsubsection{\cite{ward94}}

%  . . . . . . . . . . . . . . . . . . . . . . . . . . . . . . . . . . . . . . . . 
\subsubsection{\cite{bird10}}
\cite{bird10} using the Global Strain Rate Map (GSRM) \citep{kreemer03} prepared a 
long-term forecast of shallow seismicity.  

The GSRM is a velocity gradient tensor field. It describes the  
spatial variation of the horizontal strain rate tensor components (in diffuse plate
boundaries zones).

Also important: 
\begin{itemize}
\item \cite{kagan2010}
\end{itemize}

%
% --------------------------------------------------------------------------------
\newpage
\section{Accounting for epistemic uncertainties}
\cite{coppersmith2009} define two groups of logic-tree related variables called 
``global'' and ``local''. Global variables are the ones that affect multiple 
seismic sources such as variables controling existence and geometry of sources,
variables related to the calculation of source parameters (e.g. catalogue, 
completeness methodology, regional $b_{GR}$ values). Local variables affect only
the parameters characterizing one seismic source.
