OpenQuake computes classical PSHA 
\citep{cornell1968,mcguire2004} following the methodology proposed by 
\citet{field2003}. This methodology has the distinctive property of performing
the entire calculation using probabilities, as orginally proposed by 
\citet{chiang1984}, instead of working with occurrence rates like in most 
of the commonest PSHA codes \citep[see for instance][]{bender1987}. 
%
This methodology has the clear advantage of decoupling the creation of the 
probabilistic seismicity occurrence model (in the OpenSHA terminology this is 
defined as the Earthquake Rupture Forecast) from the assumption of a Poissonian 
temporal occurrence model. 
%
\citet{field2003} demonstrated the congruence between the original PSHA 
formulation based on occurrences and the methodology here adopted 
in the case of a Poissonian temporal model. 

From a more strict calculation perspective two, are the main steps composing the
procedure:
\begin{itemize}
\item Creation of the probabilistic seismicity occurrence model, i.e. a discrete 
distribution giving the probability of occurrence of each possible rupture 
occurring on one of the seismic sources defined in the PSHA input model in a  
given time span. This step of the procedure was already discussed within 
Chapter \ref{chap:erf} at page \pageref{chap:erf}.
\item Calculation of hazard at the site by combining the probabilistic 
occurrence model with a ground motion prediction equation (also Intensity 
Measure Relationship in the OpenSHA jargon). This second step will the topic 
of the current chapter.
\end{itemize}
%
% --------------------------------------------------------------------------------
\section{Calculation kernel}
The classical PSHA calculation kernel takes an Earthquake Rupture Forecast (ERF) 
as an input; an ERF is simply a list of all the possible ruptures occurring on 
all the seismic sources included in a Source Model; each rupture $Rup$ is  
associated with a probability of occurrence in the time span $t$ fixed for the 
analysis. 
 
%
% --------------------------------------------------------------------------------
\section{Hazard calculation: traditional formulation in terms of probabilities}
Following \cite{field2003}, the traditional formulation in terms of probabilities:
%
\begin{equation}
P(U\geq u)= 
	1-\prod\limits_{i=1}^{l} 
	\left[\sum\limits_{s=0}^{+\infty}
	\left(P(S=s) 
	\left(
		1-\sum\limits_{j=0}^{j(i)}\sum\limits_{s=0}^{K(i,j)} 
		P(m_{i,j}) 
		P(R_{i,j,k}|m_{i,j}) P(U\geq u|m_{i,j},R_{i,j,k})
	\right)
	\right)^{s}
	\right] 
\end{equation}
where $l$ i the number of sources 

If the probability of multiple occurrences is assumed to be negligible the 
probability to get an exceedance of $x$ in a given time span corresponds to:
%
\begin{equation}
P(X\geq x)=1-\prod\limits_{i=1}^{l} 
	\left( 
		1-\sum\limits_{n=1}^{N(i)}P(Rup_{i,n})P(X\geq x|Rup_{i,n})
	\right)
\end{equation}

%  - - - - - - - - - - - - - - - - - - - - - - - - - - - - - - - - - - - - - - - -
\subsection{Example}
In case of two punctual sources each one generating a single rupture, the 
probability of exceedance of ground motion $u$ in a given site corresponds to:
%
\begin{eqnarray}
P(U\geq u)=
	1-
	\biggl(& 
		\bigl[ 1-P(Rup_{i,1})P(U\geq u|Rup_{i,1}) \bigr] \nonumber \\
		& \bigl[ 1-P(Rup_{i,2})P(U\geq u|Rup_{i,2}) \bigr]
	\,\biggr)
\end{eqnarray}