In this chapter we discuss the properties of the two main calculators 
needed to process the information contained in the \gls{pshainputmodel} 
and prepare it for calculation: the Logic Tree processor and the 
Earthquake Rupture Forecast calculator. 

In Section \ref{hazard:logic_tree_processor} we describe the logic 
tree processor which takes the PSHA Input Model and creates many 
realisations of a \gls{seismicsourcemodel} and of a 
\gls{groundmotionmodel} while in the following two Sections we concentrate 
on the Earthquake Rupture Forecast Calculator which takes one seismic
sources model - produced by the Logic Tree Processor - and creates 
the \gls{acr:erf}.  
%
% ------------------------------------------------------------------------------
\section{The Logic Tree Processor}
\label{hazard:logic_tree_processor}
\index{Logic Tree!Processor}
%
The Logic Tree Processor is responsible for processing data in a 
PSHA input model describing the two logic tree structures
(see Section \ref{hazard:logic_tree}). 
%
The processing consists on producing a \gls{seismicsourcemodel} from the 
seismic source logic tree (see Section \ref{hazard:source_model_logic_tree}) 
and one \gls{groundmotionmodel} from the ground motion logic tree 
(see Section \ref{hazard:gmpe_logic_tree}). 
%
%The seismic source model and ground motion model created can be then 
%passed to the different OpenQuake-Hazard calculators: the classical PSHA 
%calculator (described 
%in Section \ref{chap:classic_psha}) and the stochastic PSHA calculator 
%(discussed in Section \ref{chap:stochastic_psha}).

The potential to process the information in a logic tree is constrained 
by its size, the complexity of the structure and the intricacy of the 
\glspl{branchset} involved.
%
For a large logic tree, performing a seismic hazard calculation for 
all possible end-branch models is an unfeasible task, therefore, 
a Monte Carlo sampling of the tree is a more efficient approach. 
%
On the contrary, for a small and simple logic tree, a Monte Carlo 
approach is inefficient with respect to enumerating all the possible 
end-branch models and performing a hazard analysis for each of them. 
%
In other words, to get stable results in case of a simple logic tree, 
a Monte Carlo approach would require sampling epistemic uncertainties 
a number of times much larger than the actual number of end-branch models.
%
The general plan for OpenQuake is to provide both the two processing 
strategies. 

Currently, the logic tree processor can only provide a Monte 
Carlo sampler.
%
\subsection{The Logic Tree Monte Carlo Sampler}
Aim of the \gls{acr:ltmcs} is to create a set of seismic source model 
and group motion model representing exhaustively the combinations 
allowed by the logic tree structures defined by the modeller.
% 
In this way the distribution of the final hazard results will reflect 
the degree of uncertainty introduced by lack of precise knowledge 
of parameters and models included in the PSHA input model.
%
\subsubsection{Sampling the source model logic tree}
As described in \ref{hazard:source_model_logic_tree}, the first branching 
level in the source model logic tree is used for defining one or more 
alternative source models called . 
Subsequent branches define parameter-specific 
epistemic uncertainties. Each branching level contains only one branch set 
(therefore producing a symmetric logic tree). 
%
The \gls{acr:ltmcs} constructs a source model by looping over all the 
branching levels. In the first branching level, a source model is randomly 
selected, with a probability equal to the 
uncertainty weight. Epistemic uncertainties defined in subsequent branching 
levels are then applied to this initially selected source model. For each 
following branching level, a loop over the seismic sources defined in the 
selected source model is started, and for each seismic source an epistemic 
uncertainty value is randomly selected (again with a probability equal to 
the uncertainty weight).
%
\subsubsection{Sampling the ground motion model logic tree}

\dotfill \newline
NOTES:
- No mention about the methodology adopted to sample the LT
- No description of correlated vs uncorrelated branches
\hfill \clearpage

