\section{The Logic Tree Processor}
\label{hazard:logic_tree_processor}
The Logic Tree Processor (hereinafter LTP) is responsible for processing data contained in a logic tree structure (see \ref{hazard:logic_tree}), that is selecting a source model from the source model logic tree (see \ref{hazard:source_model_logic_tree}) and a ground motion prediction equation from the GMPE logic tree (see \ref{hazard:gmpe_logic_tree}). The selected source model and GMPE can be then passed to the different OpenQuake calculators (Classical PSHA calculator, \ref{chap:classic_psha}; Stochastic PSHA calculator, \ref{chap:stochastic_psha}).\\
The capability to process the information contained in a logic tree is constrained by the size and complexity of the logic tree itself. For a large logic tree, performing a seismic hazard calculation for all the possible end-branch models is an unfeasible task. Under this condition, a Monte Carlo sampling of the logic tree is a more efficient approach.\\
