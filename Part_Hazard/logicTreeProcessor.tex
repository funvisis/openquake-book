The Logic Tree Processor is responsible for processing data contained in a logic tree structure (see \ref{hazard:logic_tree}), that is selecting a source model from the source model logic tree (see \ref{hazard:source_model_logic_tree}) and a ground motion prediction equation from the GMPE logic tree (see \ref{hazard:gmpe_logic_tree}). The selected source model and GMPE can be then passed to the different OpenQuake-Hazard calculators (Classical PSHA calculator, \ref{chap:classic_psha}; Stochastic PSHA calculator, \ref{chap:stochastic_psha}).\\
The capability to process the information contained in a logic tree is constrained by the size and complexity of the logic tree itself. For a large logic tree, performing a seismic hazard calculation for all possible end-branch models is an unfeasible task. Under this condition, a Monte Carlo sampling of the logic tree is a more efficient approach. On the contrary, for a small logic tree, a Monte Carlo approach is inefficient with respect to enumerating all the possible end-branch models and performing a hazard analysis for each of them. In other words, to get stable results in case of a simple logic tree,  a Monte Carlo approach would require sampling epistemic uncertainties a number of times much larger than the actual number of end-branch models.\\
The general plan for OpenQuake is to provide both the two processing strategies. Currently, the logic tree processor can only provide a Monte Carlo sampler.
\subsection{The Logic Tree Monte Carlo Sampler}
Aim of the Logic Tree Monte Carlo Sampler (hereinafter LTMCS) is to sample epistemic uncertainties, so that the distribution of the final hazard results reflect the degrees of belief that the hazard modeler specified in the different logic tree branches.\\
\subsubsection{Sampling the source model logic tree}
As described in \label{hazard:source_model_logic_tree}, the first branching level in the source model logic tree is used for defining one or more alternative source models. Subsequent branches define parameter-specific epistemic uncertainties. Each branching level contains only one branch set (therefore producing a symmetric logic tree). The LTMCS constructs a source model by looping over all the branching levels. In the first branching level, a source model is randomly selected, with a probability equal to the uncertainty weight. Epistemic uncertainties defined in subsequent branching levels are then applied to this initially selected source model. For each following branching level, a loop over the seismic sources defined in the selected source model is started, and for each seismic source an epistemic uncertainty value is randomly selected (again with a probability equal to the uncertainty weight).