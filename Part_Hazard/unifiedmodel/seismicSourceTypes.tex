An OpenQuake PSHA input model contains a number of sources belonging 
to a finite set of possible typologies. Currently OpenQuake supports 
four seismic source types. Each type contains a limited number of 
parameters necessary to specify geometry and seismicity occurrence. 

In the following sections we provide a detailed description of the 
source typologies currently supported by OpenQuake and that can be 
included in an Initial Seismic Sources Model as explained in the 
introduction of this chapter and in Section \ref{hazard:pshainputmodel}.
%
%  - - - - - - - - - - - - - - - - - - - - - - - - - - - - - - - - - - - - - - -
\subsection{Seismic source typologies description}
%
OpenQuake, at present time, provides four seismic source typologies, 
for the most part defined in the GEM1 project \citep{pagani2010}. 
%
The main seismic source types currently supported are the following:
\begin{itemize}
\item \emph{Area source} - So far, the most frequently adopted source 
type in national and regional PSHA models.
\item \emph{Grid source} - Grid sources can be considered a replacement 
for area sources since they both model distributed seismicity;
\item \emph{Simple fault source} - A Simple fault is the easiest way to
specify a fault source in OpenQuake. This typology is habitually adopted 
to describe shallow seismogenic fault sources.
\item \emph{Complex fault sources} - A complex fault is more often used 
to model subduction interface sources with a complex geometry. 
\end{itemize}

These are the basic assumptions accepted in the definition of these source 
typologies:
\begin{itemize}
\item In the case of area and fault sources, the seismicity is homogeneously 
distributed over the source; 
\item Seismicity temporal occurrence follows a Poissonian model; 
\item The frequency-magnitude distribution can be approximated to an evenly 
discretized distribution. 
\end{itemize}
%
%  . . . . . . . . . . . . . . . . . . . . . . . . . . . . . . . . . . . . . . . 
\subsubsection{Area sources}
\label{hazard:seismic_source_types:areaSources}
\index{Source type!area} 
\index{Area source|see{Source type}}
%
Area sources model the seismicity occurring over wide areas where  
identification or characterization - i.e. unambiguous definition 
of seismicity occurrence parameters - of single sources is difficult. 
%
The \citet{sshac1997} - using as a discriminant the extension - 
defined three main types of area seismic sources:
\begin{enumerate}
\item Area sources enclosing concentrated zones of seismicity;
\item Regional area sources;
\item Background area sources.
\end{enumerate}
%
The criteria adopted for their definition - and the related 
uncertainties - vary according to each area source type. 
%
From a computation standpoint we do not introduce any difference 
between these three area types.
%
%  . . . . . . . . . . . . . . . . . . . . . . . . . . . . . . . . . . . . . . . 
\paragraph{Parameters}
\begin{itemize}
\item A polygon that identifies the external border of the area. 
The current version of OQ doesn't support the definition 
of internal borders
\item One (or several) combinations of the following objects:
\begin{itemize}
	\item A discrete Frequency-Magnitude Distribution (FMD)
	\item (optional) Strike, dip, and rake angles indicating 
	main fault geometry and slip direction for the seismicity  
	in the corresponding FMD.
	%
	For example, in the PSHA model prepared within the PEGASUS project,
	\cite{coppersmith2009} defines for each area source a discrete 
	distribution of strike values (dip is not considered because the 
	source-site metrics they use is the Joyner-Boore distance). 
\end{itemize}
%
This description permits the accurate characterization of seismicity 
occurrence within an area by explicitly taking into account the existing 
faulting trends. 
%
\item An array specifying the depth to the top of rupture dependency on 
magnitude. The array contains two columns and as many columns as the 
$<$depth, magnitude$>$ tuples used. 
The depth in each tuple defines the top of rupture for magnitudes equal 
or greater than the corresponding value. 
%
\item A value to indicate the hypocentral depth in case of punctual sources. 
By convention all the events with magnitude lower than the lowest value of 
magnitude contained in the depth to the top of rupture array are modelled as 
punctual sources. 
%
On the opposite, ruptures with magnitude equal or greater than 
the lowest value of magnitude contained in the depth to the top of rupture array 
are modelled considering their finite dimensions. The finite dimension of the 
rupture is computed using a magnitude-area or magnitude-length relationship 
specified in the calculation settings file (in future versions of OQ we will 
allow the user to specify for each tectonic region the corresponding 
magnitude-scaling relationship).
\end{itemize}
%
%  . . . . . . . . . . . . . . . . . . . . . . . . . . . . . . . . . . . . . . . 
\subsubsection{Multi-depth area sources - \textcolor{red}{Currently 
implemented only in a prototype version of OQ}}
\label{hazard:seismic_source_types:multiDepthAreaSources}
\index{Source type!area multi-depth} 
\index{Multi-depth area source|see{Source type}}
Multi-depth area sources are a generalized version of area sources. This
source typology shares with the area sources most of the parameter 
necessary to specify seismicity  occurrence and geometry. 
%
Multi-depth area sources, however, allow the modeller to completely specify 
the distribution with depth of the seismicity, in terms of the top of 
rupture depth.
%
%  . . . . . . . . . . . . . . . . . . . . . . . . . . . . . . . . . . . . . . . 
\paragraph{Parameters}
\begin{itemize}
\item A polygon that identifies the external border of the area. 
The current version of OQ doesn't support the definition 
of internal borders
\item One (or several) combinations of the following objects:
\begin{itemize}
	\item A discrete Frequency-Magnitude Distribution (FMD)
	\item (optional) Strike, dip, and rake angles characterizing the 
	seismicity in the corresponding FMD.
\end{itemize}
%
This description permits the accurate characterization of seismicity 
occurrence within an area by explicitly taking into account the existing 
faulting trends. 
%
\item An array specifying the depth to the top of rupture dependency on 
magnitude. \textcolor{red}{The array contains xx columns and one or many $<$depth, magnitude$>$ 
tuples. Each tuple defines the depth to the top of rupture for magnitudes 
equal or greater than the corresponding value.}
%
\item A value of magnitude representing the threshold above which rupture 
are modelled considering their finite dimensions. The finite dimension of the 
rupture is computed using a magnitude-area or magnitude-length relationship 
specified in the calculation settings file (in future versions of OQ we will 
allow the user to specify for each tectonic region the corresponding 
magnitude-scaling relationship).
\end{itemize}
%
%  . . . . . . . . . . . . . . . . . . . . . . . . . . . . . . . . . . . . . . .
\subsubsection{Grid sources}
\index{Source type!grid}
\index{Grid source|see{Source type}}
A grid source  is a typology used to model distributed seismicity - usually 
of low and intermediate magnitude.
%
Grid sources can be considered a PSHA source model alternative to area 
sources, since they both try to represent distributed seismicity. Grid sources 
usually derive from the application of seismicity smoothing algorithms 
\citep{frankel1995,woo1996}. 
%
The use of these algorithms carries some advantages compared to area sources, 
indeed, (1) they remove most of the unavoidable degree of subjectivity due to 
the definition of the geometries and (2) they define a seismicity spatial 
pattern that is, usually, more similar to reality. Nevertheless, some smoothing 
algorithms require the a-priori definition of some setup parameters that expose 
the calculation to a certain partiality level.

Grid sources are modelled in OpenQuake simply as a set of 
point sources. The next section describes the parameters required to 
characterize a point source.
%
%  . . . . . . . . . . . . . . . . . . . . . . . . . . . . . . . . . . . . . . . 
\paragraph{Parameters}
%
For each grid node:
\begin{itemize}
\item A location specified in terms of the $<$latitude,longitude$>$ tuple;
\item Similarly to area sources, one (or many) combinations of the following 
objects:
	\begin{itemize}
	\item A discrete Frequency-Magnitude Distribution (FMD)
	\item Strike, dip, and rake angles characterizing the seismicity 
	specified in the associated FMD. 
	\end{itemize}
\item An array to specify the dependency on magnitude of the depth to 
	the top of rupture. This array contains two columns and one or many 
	$<$depth, magnitude$>$ tuples where each tuple specifies the depth to the 
	top of rupture for magnitudes equal or greater than a specific value. 
\item A value to indicate the hypocentral depth in case of punctual sources. 
	The same convention specified for area sources applies here. 
\end{itemize} 
%
%  . . . . . . . . . . . . . . . . . . . . . . . . . . . . . . . . . . . . . . .
\subsubsection{Simple faults}
\index{Source type!fault!simple geometry} 
\index{Simple fault|see{Source type}}
%
Simple Faults are the most common source type used to model faults; the 
``simple'' adjective relates to the geometry description of the source 
which is basically obtained by projecting a trace (i.e. a polyline) along 
a representative dip direction. 
%
%  .   .   .   .   .   .   .   .   .   .   .   .   .   .   .   .   .   .   .   . 
\paragraph{Parameters}
%
\begin{itemize}
\item A fault trace (usually a polyline); 
\item A FMD;
\item A representative value of the dip angle (specified according to 
the Aki-Richards convention; see \citet{aki2002});
\item Rake angle (specified following the Aki-Richards convention; 
see \citet{aki2002}) 
\item Upper and lower values of depth limiting the seismogenic interval 
\item A boolean flag that specifies if the size of ruptures should 
follow a magnitude scaling relationship (currently specified in the 
calculation settings file) and be distributed homogeneously over the 
fault surface or it is accepted that ruptures within a given range of 
magnitudes (specified by the FMD) will always rupture the entire fault 
surface.
\end{itemize}
%
%  . . . . . . . . . . . . . . . . . . . . . . . . . . . . . . . . . . . . . . .
\subsubsection{Complex faults}
\index{Source type!fault!complex geometry}
\index{Complex fault|see{Source type}}
%
Complex faults  differ from simple fault just by the way geometry is 
described and, consequently in the way the fault surface is created. The 
input parameters used to describe complex faults are, for the most part, 
the same used to describe the simple fault typology. In particular, in 
the case of complex faults the dip angle is not requested while the fault
trace is substituted by two fault traces used to limit at top and bottom 
the fault surface. 
%
Usually, we use complex faults to model intraplate megathrust faults such 
as the big subduction structures active in the Pacific (Sumatra, South 
America, Japan).
