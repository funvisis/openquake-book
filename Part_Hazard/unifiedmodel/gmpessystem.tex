% Table: list of GMPEs currently supported by OQ
\input{./Part_Hazard/tables/gmpeslist.tex}
%
The Ground Motion System is a combination of one or several logic 
trees each one associated with a specific tectonic region or group 
of sources (this second option is still not supported in OQ).
%
Each Ground Motion Logic Tree specifies the alternative Ground Motion 
models available for a particular group of sources (e.g. subduction 
interface sources).

OpenQuake provides only hardcoded \gls{acr:gmpe} 
and misses of a mechanisms allowing the user to specify new GMPEs. 
This is a feature that we may think to introduce in the future. 

Table \ref{tab:OQ_GMPEs} provides a list of the Ground Motion Prediction 
Equations supported. 

The vast majority are GMPEs implemented in OpenSHA with just a couple 
of developed in the course of the GEM1 project. New GMPEs are expected 
to be added soon with the contribution of some GEM's Regional Programmes.

%  - - - - - - - - - - - - - - - - - - - - - - - - - - - - - - - - - - - - - - -
\subsubsection{Ground Motion Logic Tree}
\label{hazard:gmpe_logic_tree}
\index{Logic Tree!Ground Motion}
The Ground Motion Logic Tree accounts for epistemic uncertainties related 
to the Ground Motion models.  
%
Given that ground motion models are often, or can be, associated to 
specific tectonic region,  OpenQuake supports the definition of multiple 
GMPE logic trees, one for each tectonic region type considered in 
the source model. 
% 
For example, if a PSHA Input model contains seismic sources belonging to 


In the current version, a GMPE logic tree can have only one 
branching level, containing only one branch set, where each individual branch 
is associated to a specific GMPE. With the current setting, epistemic 
uncertainties coming from different models can be taken into account, but 
epistemic uncertainties inside each model cannot be captured.
Figure \ref{fig:GMPELogicTree} schematically shows GMPE logic trees that can 
be currently defined in OpenQuake.
% ..............................................................................
% . . . . . . . . . . . . . . . . . . . . . . . . . . . . . . . . . . . > Figure 
\renewcommand{\psedge}{\ncdiag[armA=0,angleB=180,armB=1cm]}
\begin{figure}[!hb]
\centering
\input{./Part_Hazard/pstricks/logicTreeGmpes.tex}
\caption{Examples of Ground Motion Logic Trees. The first specifies ground
motion models for active shallow crust (in this case we consider three GMPEs) 
the second defines the ground motion models to adopt for subduction interface 
sources (in the example depicted above we consider two GMPEs).}
\label{fig:GMPELogicTree}
\end{figure}

% . . . . . . . . . . . . . . . . . . . . . . . . . . . . . . . . . . . < Figure
% ..............................................................................
