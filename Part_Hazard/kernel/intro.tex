OpenQuake computes probabilistic seismic hazard using two different 
methodologies: the classical one and a methodology based on the 
generation of a stochastic event set.

The classical PSHA methodology adopted in \gls{acr:oq} is the one 
presented by \citet{field2003} and implemented in OpenSHA. 
This particular formulation - as also proposed by \citet{chiang1984} - 
has the distinctive property of performing the entire calculation using 
probabilities instead of working with occurrence rates like in most 
of the commonest PSHA codes \citep[see for instance][]{bender1987}. 
%
The \gls{opensha} methodology has also the clear advantage of decoupling the 
creation of the probabilistic seismicity occurrence model (in the OpenSHA 
terminology this is defined as the \gls{earthquakeruptureforecast}) from the 
assumption of a Poissonian temporal occurrence model. 
%
\citet{field2003} demonstrated that by assuming negligible contributions to 
hazard coming from multiple occurrences - i.e. the probability that a source 
will generate two or more occurrences of the same rupture within the time 
span fixed for the analysis is equal to zero - this methodology is 
completely consistent with the most classical procedure.

The \gls{acr:oq} stochastic event based PSHA calculation procedure 
resembles recent approaches proposed in the literature (see for example 
\citet{musson2000} and references therein). 
%
The major advantages of this approach are that (1) hazard can be directly 
linked to a sequence of earthquakes and (2) the residuals of ground-motion 
on each investigated site can be distinctly considered, eventually by 
taking into account the spatial correlation of ground-motion.

The two hazard calculation kernels have in common a first calculation 
phase (see also Figures \ref{classical_psha_workflow} and 
\ref{event_based_workflow}) which is the creation of the 
\gls{earthquakeruptureforecast}. 
The creation of the \gls{acr:erf} was extensively discussed in Chapter 
\ref{chap:erf} at page \pageref{chap:erf}.
%
The second phase in the classical PSHA approach corresponds to the 
calculation of hazard at the site by combining the probabilistic 
seismicity occurrence model with a ground-motion model. 
It will be discussed in section (\ref{chap:classic_psha} 
at page \pageref{chap:classic_psha}).
%
The second calculation step in the event based PSHA approach is the 
generation of the stochastic event set, the calculation of the 
corresponding ground-motion fields and, eventually, the post-processing 
of the set of ground-motion field computed to obtain the final values 
of hazard. This methodology will be more extensively described in section 
\ref{chap:stochastic_psha} at page \pageref{chap:stochastic_psha}.
%
The last section of this Chapter (see page \pageref{sec:disaggregation}) 
is dedicated to the discussion of the disaggregation methodology implemented 
in OQ.
