%-------------------------------------------------------------------------------
\section{Classical PSHA calculator}
\label{chap:classic_psha}
%
This calculation methodology is the one we consider the most efficient 
for the calculation of traditional PSHA results such as hazard maps,
hazard curves and uniform hazard spectra.
%
The classical PSHA calculation kernel \citep{field2003} takes as 
input the following information: 
%
\begin{itemize}
\item An \gls{earthquakeruptureforecast} 
\item A \gls{groundmotionmodel} 
\end{itemize}
%
%- - - - - - - - - - - - - - - - - - - - - - - - - - - - - - - - - - - - - - - -
\subsection{PSHA calculation: assuming a negligible contribution from 
repeated ruptures in $t$}
%
The classical PSHA calculation methodology available in OpenQuake 
is the one currently implemented in \gls{opensha}; this methodology
works in terms of probabilities instead of occurrences as many of 
the currently available PSHA codes currently do. 
%
As demonstrated by \citet{field2003}, this method is coherent 
with the classical one under the assumption that the contribution 
to hazard coming from multiple ruptures is negligible (i.e. 
the probability that $src$ will repeat two - or more - $rup$ in $t$ 
is very very low). 

The calculation of hazard for a single site $site$ and a single 
ground-motion parameter $y$ simply consists of an iterative procedure 
that integrates the contributions coming from the ruptures included in 
the \gls{acr:erf} and located at a distance from the $site$ shorter 
than a threshold value (often set in the range 200 to 300 km). 
%
During each iteration, OQ takes a rupture $rup$ within source $src$ and
calculates the probability of exceedance of $y$ in a investigation in 
time $t$ at $site$ using the following equation:
\begin{equation}
P(Y \geq y|t,rup_{src},site) = 
	P(rup_{src}|t)\,
	P(Y\geq y|rup_{src},site)
\label{eq:prob_y_ex_one_rup}
\end{equation}
\marginpar{Check if this equation is formally correct}
The probability $P(Y \geq y|t,rup_{src},site)$ corresponds to 
the product between the probability of occurrence of $rup$ in a time 
$t$ and the conditional probability of exceeding $y$ at $site$ 
given the occurrence of $rup$. 
%
This conditional probability is usually computed by means of a 
\gls{groundmotionpredictioneq} which provides, given a rupture 
and a site, the first two moments of a gaussian distribution 
(a general and common assumption in the ground-motion community). 
\marginpar{here we need a citation}
%
On the contrary, $P(rup_{src}|t)$ is the probability of occurrence 
attributed to $rup$ during the creation of the 
\gls{earthquakeruptureforecast}.
Equation \ref{eq:prob_y_ex_one_rup} can also be rewritten by 
substituting to each rupture the corresponding magnitude and 
node within source $src$.
\begin{equation}
P(Y \geq y|t,m,node_{src},site) = 
	P(m,node|t)\,
	P(Y\geq y|m,node_{src},site)
\end{equation}
This equation states that the probability of exceedance of $y$ in $t$
corresponds to the product between the probability of occurrence
of magnitude $m$ on node $node$ (in OpenSHA and OQ sources are always 
discretized in a number of nodes) and the probability of exceedance of
$y$ given the occurrence of $m$ on $node$. 

Assuming that ruptures within a source are mutually exclusive, 
the probability $P(Y\geq y|t,src,site)$ that at least one rupture 
generated by source $src$ will produce during the investigation 
time $t$ at least one exceedance of $y$, corresponds to the 
difference between unity and the probability that none of the 
ruptures will generate an exceedence of $y$ in $t$.
\begin{equation}
P(Y\geq y|t,src,site) = 1 - \sum_{\forall\,rup\,\text{in}\,src}^{} 
	\Big( P(rup_{src}|t)\,
	P(Y\geq y|rup_{src},site) \Big)
\label{eq:class_psha_1}
\end{equation}
The final value of hazard at site $site$ will be obtained by 
merging the contributions coming from the totality of sources 
considered during the creation of the \gls{acr:erf}, under the 
assumption that events occurring within different sources are 
independent.
%
\begin{equation}
P(Y \geq y|t,site) = 1 - \prod_{\forall\,src\,\text{in}\,ERF}^{} 
\Big( 1-P(Y\geq y|t,src,site) \Big) \label{eq:class_psha_2}
\end{equation}
%
Combining equations \ref{eq:class_psha_1} and \ref{eq:class_psha_2} 
we get \cite[see also][equation 4, page 410]{field2003}:
%
\begin{multline}
P(Y \geq y|t,site) = \\
	1-\prod\limits_{\forall\,src\,\text{in}\,ERF}^{}  
	\left\{
		1-\sum\limits_{\forall\,rup\,\text{in}\,src}^{} 
			\biggl[ P(rup_{src}|t)\,P(Y\geq y|rup_{src},site)
			\biggr]
	\right\}
\label{eq:PSHA_calc_classical_no_repeating}
\end{multline}
%
%- - - - - - - - - - - - - - - - - - - - - - - - - - - - - - - - - - - - - - - -
\subsection{PSHA calculation: accounting for contributions from 
repeated ruptures in $t$}
%
Sometimes  (e.g. in case of particular time dependent PSHAs) the 
assumption of negligible contributions to the final value 
of hazard coming from repeated ruptures does not hold anymore (e.g. in 
case of particular time dependent PSHAs). 
%
Consequently, for precise hazard calculations is necessary to take 
into account any possible contribution produced by the sources in the
\gls{earthquakeruptureforecast}.

In particular, in order to account for repeated ruptures equation 
\ref{eq:prob_y_ex_one_rup} must be rewritten as 
\begin{multline}
P(Y \geq y|t,rup_{src}^{*},site) = \\
	\sum\limits_{n=1}^{\infty}
	\bigl( {1 - P(Y\geq y|rup_{src},site)^{n}}\bigr)
	P(\#rup_{src}=n|t)\,
\label{eq:prob_y_ex_many_rup}
\end{multline}
where $P(Y \geq y|t,rup_{src}^*,site)$ stands for the 
probability of at least one exceedance of $y$ given one or several 
ruptures $rup_{src}$ occurring within source $src$.
% 
As a result, equation \ref{eq:class_psha_1} becomes 
\begin{multline}
P(Y \geq y|t,src,site) = \\
1 - \sum_{\forall\,rup\,\in\,src}^{} 
	\biggl(\,
	\sum\limits_{n=1}^{\infty}
	1 - P(Y\geq y|rup_{src},site))^{n}\bigr)
	P(\#rup_{src}=n|t)
	\biggr)
\end{multline}
%
Substituting equation \ref{eq:prob_y_ex_many_rup} into equation 
\ref{eq:PSHA_calc_classical_no_repeating}
\begin{multline}
P(Y \geq y|t,site) = \\
	1-\prod\limits_{\forall\,src\,\text{in}\,ERF}^{}
	\left\{
		1-\sum\limits_{\forall\,rup\,\text{in}\,src}^{} 
			\biggl[ P(rup|t)\,P(Y\geq y|rup_{src},site)
			\biggr]
	\right\}
\end{multline}
\marginpar{this equation must be updated!}
