%
%-------------------------------------------------------------------------------
\clearpage\newpage
\section{PSHA disaggregation}
\label{sec:disaggregation}
%
Seismic hazard disaggregation - or deaggregation - 
\citep{mcguire1995,bazzurro1999} is a procedure aimed at identifying the 
contributions to a specified level of hazard coming from different combinations
of basic variables - such as magnitude and rupture-site distance - 
characterizing the ruptures included in the ERF.
%
Two are the main typologies of disaggregation currently adopted in PSHA 
studies (see for example \citet{petersen2008}): the M-R-$\epsilon$ 
disaggregation and the geographic disaggregation. Conceptually there 
are no differences between the two; simply we can observe that in the 
geographic disaggregation the source-to-rupture distance is replaced by
the position on the topographic surface of the rupture-point used to 
calculate the distance to the site.
%

\subsection{M-D disaggregation}
We start the description of the disaggregation procedure by considering 
the  simplest disaggregation case. 

As a first step we create a container where to progressively store the 
contributions coming from the different ruptures produced by the seismic 
sources contained in the Seismic Source Model.
%
To this purpose, we use a two-dimensional matrix $Q$ with a number or 
rows equal to the number of discrete distance bins and a number of 
columns equal to the number of discrete magnitude intervals. 

The magnitude and distance ranges used to create the $Q$ matrix cover 
the spectrum of values used in the hazard calculation. In particular, 
the magnitude range can be determined considering the properties of 
the seismicity occurrence model for each of the sources contained in
the Seismic Source Model whilst the distance range is implicitly defined
by a maximum integration distance parameter to be specified in the OQ 
configuration file. 
%

The argument of the summation contained in Equation \ref{eq:PSHA_calculation} 
computes the probability of exceedance in a time $t$ of $gm$ generated by the 
occurrence of the rupture $Rup$. Without loss of generality a rupture can be
associated to a magnitude and a source-to-site distance.