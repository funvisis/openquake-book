%
%-------------------------------------------------------------------------------
\clearpage\newpage
\section{PSHA disaggregation}
\label{sec:disaggregation}
%
Seismic hazard disaggregation - or deaggregation - 
\citep{mcguire1995,bazzurro1999} is a procedure aimed at identifying the 
contributions to a specified level of hazard -  referred to a specific site -
coming from different combinations of basic variables - such as magnitude
and rupture-site distance - characterizing the ruptures included in the 
\gls{acr:erf} and the ground motion models selected.
%
Two are the main typologies of disaggregation currently adopted in PSHA 
studies (see for example \citet{petersen2008}): the M-R-$\epsilon$ 
disaggregation and the geographic disaggregation. 
%
Conceptually there are no differences between them; 
simply, in the geographic disaggregation the source-to-rupture 
distance is replaced by the position of the point used to calculate
the distance to the site.
%

In \gls{acr:oq} we introduced a number of new disaggregation typologies 
that should help the modeller in better understanding the PSHA models   
and controlling the results provided for specific sites.
% 
In particular, we added tectonic region and seismic source type to 
the classical disaggregation variables (i.e. magnitude, distance and
epsilon); this way is possible to clearly identify the contribution to 
hazard coming from distinct variables and source/ground motion model
attributes.

The disaggregation methodologies implemented in \gls{acr:oq} is flexible 
enough to provide the best combination possible according to the 
modeller's needs.
% 
\gls{acr:oq} supports disaggregation based on the classical PSHA 
methodology as well as on the event based methodology one. In the 
following sections we describe the details of the approaches
implemented.
%
%- - - - - - - - - - - - - - - - - - - - - - - - - - - - - - - - - - - - - - - -
\subsection{Disaggregation methodology}
Disaggregation from a calculation point of view, simply consists on 
systematically collecting the contributions (i.e. conditional probabilities
of exceedance) to a selected value of hazard for a specific site coming 
from all the rupture included in an \gls{acr:erf}. 
%
In \gls{acr:oq} we use a multidimensional matrix (called disaggregation
matrix) to store these contributions; this matrix contains indexes for:
\begin{itemize}
\item Magnitude (index i)
\item Longitude (index j)
\item Latitude (index k)
\item Source (index s)
\item Epsilon (index l)
\end{itemize}
%
For each rupture the longitude and latitude coordinates is the point 
on the rupture used to calculate the source-site distance.
%
%. . . . . . . . . . . . . . . . . . . . . . . . . . . . . . . . . . . . . . . .
\subsubsection{Classical PSHA: examples of application of the 
disaggregation methodology}
%
In case of the classical PSHA methodology, the cumulation in 
the disaggregation matrix of contributions is based on a 
somewhat modified version of Equation \ref{eq:prob_gm_ex_one_rup}
that distinctly accounts for inputs to the probability of exceedance
of $gm$ coming from different $\epsilon$ intervals.
%
\begin{equation}
P(GM \geq gm|t,rup_{src},\epsilon,site) = 
	P(rup_{src}|t)\,
	P(GM\geq gm|rup_{src},\epsilon,site)
\label{eq:prob_gm_ex_one_rup_eps}
\end{equation}
%
This equation is used recursively to store in the appropriate 
cell the contribution coming from all the ruptures included in 
\gls{acr:erf}. 
%
%
\paragraph{Disaggregation in terms of tectonic region}
The disaggregation in terms of tectonic region is probably the simplest 
disaggregation from a conceptual point of view. 
\begin{multline}
P(GM \geq gm|t,site) = \\
	1-\prod\limits_{\forall\,src\,\text{in}\,ERF}^{}  
	\left\{
		1-\sum\limits_{\forall\,rup\,\text{in}\,src}^{}
		\biggl[ 
			1-\sum\limits_{\forall\,\epsilon}^{} 
			P(GM \geq gm|t,rup_{src},\epsilon,site)
		\biggr]
	\right\}
\label{eq:disaggregation_kernel}
\end{multline}
%

%
%
\paragraph{Disaggregation in terms of tectonic region, magnitude,
distance and epsilon}


%
%. . . . . . . . . . . . . . . . . . . . . . . . . . . . . . . . . . . . . . . .
\subsubsection{Event based PSHA: examples of application of the 
disaggregation methodology}
%
In case of the event-based PSHA methodology the disaggregation consist 
on the cumulation of the events with specific characteristics. 
