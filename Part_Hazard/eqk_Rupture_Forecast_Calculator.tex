\index{Source model}
\index{Source model!Inital}
The Source System is an OpenQuake object that contains the information necessary to create a Source Model, eventually by taking into account the epistemic uncertainties. 
%
In particular, the Source System contains:
\begin{itemize}
\item One or several initial source models;
\item One logic tree describing epistemic uncertainties connected with the objects and parameters characterizing the initial source models.
\end{itemize}
%
The creation of a Source Model in OpenQuake is done by a calculator, named Logic Tree processor, that uses one of the initial source models included  and the logic tree structure in the Source System. 
% 
Once a Source Model is created the Earthquake Rupture Forecast calculator can be used to obtain a list of ruptures that can be generated by all the sources included in the Source Model. Each rupture is associated with a probability of occurrence in the time span specified in the calculation settings. 
 


Two are the main components of OpenQuake necessary to create an ERF

The Earthquake Rupture Forecast calculator creates  computes for each rupture the probability of occurence in a time span specified in the configuration file.
The list of ruptures correponds to all the possible ruptures that can be generated by the all the seismic sources included in \textbf{Source Model} can generate. Each rupture has 


%
% ------------------------------------------------------------------------------
\section{The Logic Tree processor}
Logic Tree processor

%
% ------------------------------------------------------------------------------
\section{ERF calculator}

%
%  - - - - - - - - - - - - - - - - - - - - - - - - - - - - - - - - - - - - - - -
\subsection{ERF creation in case of Grid sources}
Area sources (see also Section \ref{hazard:seismic_source_types:areaSources} 
at page \pageref{hazard:seismic_source_types:areaSources}) 

%
%  - - - - - - - - - - - - - - - - - - - - - - - - - - - - - - - - - - - - - - -
\subsection{ERF creation in case of Grid sources}

%
%  - - - - - - - - - - - - - - - - - - - - - - - - - - - - - - - - - - - - - - -
\subsection{ERF creation in case of Fault sources}

%
%  . . . . . . . . . . . . . . . . . . . . . . . . . . . . . . . . . . . . . . .
\subsubsection{Fault sources with simple geometry}

%
%  . . . . . . . . . . . . . . . . . . . . . . . . . . . . . . . . . . . . . . .
\subsubsection{Fault sources with complex geometry}
