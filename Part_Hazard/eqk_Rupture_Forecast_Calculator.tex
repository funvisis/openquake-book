\index{Source Model}
\index{Source Model!Inital (ISM)}
%
The calculation of the Seismicity Occurrence Model (i.e. ERF) starts from a Source model. 
%
The creation of a Source Model in OpenQuake is done by a calculator, named Logic Tree Processor (LTP). The LTP uses one of the ISMs and the logic tree structure included in the Source System to create Source Model.
%
A Source Model is a complete description of the type, geometry and seismicity occurrence properties of the seismic sources necessary to calculate comprehensively the hazard over the investigated area. In a Source Model is not possible to inc epistemic uncertainties.

Once a Source Model is created, using the Earthquake Rupture Forecast calculator we create a list of the ruptures that can be generated by all the sources included in the Source Model. Each rupture is associated with a probability of occurrence in the time span specified in the calculation settings. 
 

The Earthquake Rupture Forecast calculator creates  computes for each rupture the probability of occurence in a time span specified in the configuration file.
The list of ruptures correponds to all the possible ruptures that can be generated by the all the seismic sources included in Source Model can generate.
%
Each rupture has 


%
% ------------------------------------------------------------------------------
\section{The Logic Tree processor}
Logic Tree processor

%
% ------------------------------------------------------------------------------
\section{ERF calculator}

%
%  - - - - - - - - - - - - - - - - - - - - - - - - - - - - - - - - - - - - - - -
\subsection{ERF creation in case of Grid sources}
Area sources (see also Section \ref{hazard:seismic_source_types:areaSources} 
at page \pageref{hazard:seismic_source_types:areaSources}) 

%
%  - - - - - - - - - - - - - - - - - - - - - - - - - - - - - - - - - - - - - - -
\subsection{ERF creation in case of Grid sources}

%
%  - - - - - - - - - - - - - - - - - - - - - - - - - - - - - - - - - - - - - - -
\subsection{ERF creation in case of Fault sources}

%
%  . . . . . . . . . . . . . . . . . . . . . . . . . . . . . . . . . . . . . . .
\subsubsection{Fault sources with simple geometry}

%
%  . . . . . . . . . . . . . . . . . . . . . . . . . . . . . . . . . . . . . . .
\subsubsection{Fault sources with complex geometry}
