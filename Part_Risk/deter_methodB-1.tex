\section{Method B}
\subsection{Description}
In this approach, for each ground motion field, the intensity measure level at a given site is used to calculate the mean and standard deviation of loss ratio using the vulnerability functions for each asset contained in the exposure file. Using these results, the mean and standard deviation of loss ratio across all events can be calculated. Again, loss ratios are converted into losses by multiplying by the value of the asset given in the exposure file. For this method, it is possible to aggregate the losses throughout the region and to compute the standard deviation of the aggregated loss. 

\subsection{Calculation workflow}

To compute the mean loss:

\begin{enumerate}
\item For each ground motion field, the intensity measure levels are related with the vulnerability functions to compute the mean loss ratio for each asset. Since currently the vulnerability functions are being defined in a discrete way, it is quite probable that the intensity measure level provided by the ground motion field is not contained in the vulnerability function. In these cases, linear interpolation methods are being employed to derive the mean loss ratio at the intensity measure level of interest.

\item The mean loss ratio for each asset across all possible simulations of the deterministic event can be calculated through the formula:

\begin{equation}
LR=\frac{\sum^m_{n=1}LR_n|IML}{m}
\end{equation}

Where $m$ stands for the number of ground motion fields simulated.

\item The mean loss can then be derived by multiplying the mean loss ratio by the value of the asset contained in the exposure model file.

\end{enumerate}

To compute the standard deviation of the loss:

\begin{enumerate}

\item Again, the total probability theorem is employed. $E[LR_n^2]$ is computed using the following formula:

\begin{equation}
E[LR_n^2]=SD[LR_n]^2+E[LR_n]^2
\end{equation}

Where $SD[LR_n]$ stands for the standard deviation of the distribution of loss ratios and $E[LR_n]$ stands for the mean loss ratio (for each ground motion field).

\item Then, the total $E[LR^2]$ can be derived using the formula:

\begin{equation}
E[LR^2]=\frac{\sum_{n=1}^m E[LR_n]^2}{m}
\end{equation}

\item The standard deviation of the loss ratio can be computed using the expression:

\begin{equation}
SD[LR]=\sqrt{E[LR^2]-E[LR]^2}
\end{equation}

Where $E[LR]$ stands for the mean loss ratio computed previously.

\item The standard deviation of the absolute loss can finally be computed by multiplying the standard deviation of the loss ratio by the value of the respective asset.

\end{enumerate}